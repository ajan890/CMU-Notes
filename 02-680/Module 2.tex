% document formatting
\documentclass[10pt]{article}
\usepackage[utf8]{inputenc}
\usepackage[left=1in,right=1in,top=1in,bottom=1in]{geometry}
\usepackage[T1]{fontenc}
\usepackage{xcolor}

% math symbols, etc.
\usepackage{amsmath, amsfonts, amssymb, amsthm}

% lists
\usepackage{enumerate}

% images
\usepackage{graphicx} % for images
\usepackage{tikz}

% code blocks
% \usepackage{minted, listings} 

% verbatim greek
\usepackage{alphabeta}

\graphicspath{{./assets/images/Module 2}}

\title{02-680 Module 2 \\ \large{Essentials of Mathematics and Statistics}}
 
\author{Aidan Jan}

\date{\today}

\begin{document}
\maketitle

\section*{Number Sets}
\subsection*{Booleans}
A boolean is a data type that takes two values: True and False.
\begin{itemize}
	\item Sometimes we use 1 and 0 instead
	\item Can also be thought of as yes/no, on/off, etc.
	\item can be represented using 1 bit.
\end{itemize}

\subsection*{Integers}
An integer is a number with \textbf{no fractional} part.
\begin{itemize}
	\item We use the symbol $\mathbb{Z}$ to represent the set of \textbf{all possible integers}
	\begin{itemize}
	    \item There is a lot of them (infinitely many!), so unlike booleans we can't just list all integers.
	    \item Can be positive or negative or zero. (As opposed to \textbf{natural numbers}, represented by $\mathbb{N}$, and includes only positive integers)
	    \item The number of bits needed to represent an integer increases with the value.
    \end{itemize}
\end{itemize}

\subsection*{Rational Numbers}
A rational number is one that can be represented by a \textbf{ratio} between two integers.  That is, it can be written in form $\frac{m}{n}$, where $m, n \in \mathbb{Z}$.
\begin{itemize}
	\item We use the symbol $\mathbb{Q}$ to represent the set of all \textbf{rational numbers}.
	\item Real numbers that are not rational are called \textbf{irrational numbers}.
\end{itemize}

\subsection*{Real Numbers}
A real number is one that can be used to \textbf{measure a continuous one-dimensional quantity} such as a length, duration, or temperature.
\begin{itemize}
	\item Includes all integers, rational numbers, and all numbers "between" rational numbers.
	\item We use the symbol $\mathbb{R}$ to represent the set of \textbf{all real numbers}.
	\item Typically we need more bits to store real-valued numbers
	\begin{itemize}
	    \item proportional to the value and precision
	    \item many representations
    \end{itemize}
\end{itemize}

\subsection*{Complex Numbers}
Complex numbers were invented in order to allow for taking the square root of negative numbers
\begin{itemize}
	\item Each complex number has a \textbf{real} and \textbf{imaginary} part, and is written as $a + bi$, where $a, b \in \mathbb{R}$
    \item We denote the set of all \textbf{complex numbers} as $\mathbb{C}$.
    \item Note that $\mathbb{R}$ is a subset of $\mathbb{C}$, where all values of $b$ are 0.
\end{itemize}

\section*{Set Theory}
Why does set theory matter?\\\\
Suppose we define a property like:
\begin{center}
    "Genes that are significantly upregulated under condition $A$"
\end{center}
Then we collect all the genes that match this property.  This collection is a \textbf{set}.\\\\
We may also define a set by listing its members.  Single-cell transcriptomics may identify:
\begin{center}
    Cell type set = \{T cell, B cell, Macrophage, Fibroblast\}
\end{center}
A \textbf{sample space} is the \textbf{collection} of all possible outcomes of an experiment.
\begin{itemize}
	\item The sample space of an experiment can be thought of as a set, or collection, of different possible outcomes; and each outcome can be thought of as a point, or an element, in the sample space.
\end{itemize}

\subsection*{Venn Diagrams}
Venn diagrams can be used to compare sets.  Each circle represents a set, and members common to multiple sets are included in the overlap of the circles.\\\\
Below is an example using differentially expressed genes (DEGs):
\begin{center} 
	\includegraphics*[width=0.5\textwidth]{M2_1.png} 
\end{center}

\subsection*{What is a Set?}
A \textbf{set} is an \textbf{unordered collection} of objects.  We have already been talking about these in the abstract:
\begin{itemize}
	\item Set of all integers $\mathbb{Z}$
	\item Set of all real numbers $\mathbb{R}$
\end{itemize}
We write a set using braces "\{" and "\}"
\begin{itemize}
	\item use capital letters as names (when applicable)
	\item if we wanted to define the set of all algebraic operators we may use $O = \{+, -, \cdot, \div\}$
	\item or maybe the set of all prime numbers: $P = \{2, 3, 5, 7, \dots\}$
\end{itemize}
We have two standard ways of explicitly defining a set.
\begin{itemize}
	\item First is simple (exhaustive) enumeration:
	\[A = \{\text{"Welcome"}, \text{"to"}, \text{"02-680"}\}\]
    In this case the set contains 3 elements, each of which is a string.
    \item The second way is to define a set by abstraction:
    \[B = \{x^2 | x = 2 \lor x = 3\}\]
    We could say that $4 \in B$, but $16 \notin B$.
    \begin{itemize}
	    \item The definition after the $\vert$ is a statement written in \textbf{propositional logic}.
    \end{itemize}
\end{itemize}

\subsection*{Set Membership}
For a set $S$ and an object $x$, the expression $x \in S$ is true when $x$ is one of the objects in the set $S$.
\begin{itemize}
	\item We read $x \in S$ as "$x$ is an element of $S$", or "$x$ is in $S$".
	\item There is also the notion of non-membership
	\begin{itemize}
	    \item if the set $P$ is the set of all primes, then $4 \notin P$.
	    \item that is, $4$ is not in $P$, or $4$ is not prime.
    \end{itemize}
\end{itemize}

\subsection*{Set Cardinality}
For a set $S$, $|S|$ is the number of distinct elements in $S$.\\\\
Often we want to know how large a set is so we can compare sets.
\begin{itemize}
	\item For the set of all algebraic operators described before $|O| = 4$
	\item For the set of all possible bit values $B$, $|B| = 2$.
\end{itemize}
but sometimes, we can't actually count how many there are.  For example,
\begin{itemize}
	\item $|\mathbb{Z}| = \infty$
	\item but, we still know that $|\mathbb{Z}| < |\mathbb{R}|$, since we know that $\mathbb{Z}$ is a subset of $\mathbb{R}$.
\end{itemize}
In the examples mentioned earlier:
\[A = \{\text{"Welcome"}, \text{"to"}, \text{"02-680"}\}\]
\[B = \{x^2 | x = 2 \lor x = 3\}\]
$|A| = 3$ and $|B| = 2$.\\\\
Note that if we define another set:
\[C = \{(2 + 2), (2 - 2), (2 \cdot 2), (2 \div 2), (2^2)\}\]
the cardinality $|C| = 3$.  It is not $5$ since cardinality only includes unique elements.

\subsection*{Set Construction}
As described earlier, there are two ways to construct a set.  Exhaustive enumeration and set abstraction.
\begin{itemize}
	\item Note that not all sets need to be assigned a name.
	\item Remember that sets are \textbf{unordered}, so the following sets are equivalent.
	\begin{itemize}
	    \item $\{2 + 2, 2 \cdot 2, 2 \div 2, 2 - 2\}$
	    \item $\{0, 1, 4\}$
	    \item $\{4, 0, 1\}$
    \end{itemize}
\end{itemize}

\subsubsection*{Set Abstraction}
Let $U$ be a set of possible elements called the \textbf{universe}.  Let $P(x)$ be a condition, also called a \textbf{predicate}, that
\begin{itemize}
	\item for every element $x \in U$
	\item $P(x)$ is \textbf{true} or \textbf{false}.
\end{itemize}
then we can write $\{x \in U \::\: P(x)\}$
\begin{itemize}
	\item which is all of the objects from $x \in U$
	\item for which $P(x)$ is \textbf{true}.
\end{itemize}
For example:
\begin{itemize}
	\item set of even prime numbers: $\{x \in \mathbb{Z} \geq 1\::\: \text{$x$ is prime and $x$ is even}\}$
	\item set of primes between 10 and 20: $\{y \in \mathbb{Z} \::\: \text{$y$ is prime and $10 \leq y \leq 20$}\}$
	\item set of bits: $\{b \in \mathbb{Z} \::\: b^2 = b\}$
\end{itemize}
For convenience, sometimes we don't use the universe if it's obvious
\begin{itemize}
	\item this is particularly helpful when the predicate determines the universe
	\item In that case we could write something like $\{x \::\: P(x)\}$
\end{itemize}
There is, of course, multiple ways to write these sets as well.
\begin{itemize}
	\item Two digit perfect squares: $\{n \in \mathbb{Z} \::\: \sqrt{n} \in \mathbb{Z} \text{ and } 10 \leq n \leq 99\}$
	\item $\{n^2 \::\: n \in \mathbb{Z} \text{ and } 10 \leq n^2 \leq 99\}$
	\item $\{n^2 \::\: n \in \{4, 5, 6, 7, 8, 9\}\}$ are the same set
\end{itemize}

\end{document}