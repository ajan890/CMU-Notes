% document formatting
\documentclass[10pt]{article}
\usepackage[utf8]{inputenc}
\usepackage[left=1in,right=1in,top=1in,bottom=1in]{geometry}
\usepackage[T1]{fontenc}
\usepackage{xcolor}

% math symbols, etc.
\usepackage{amsmath, amsfonts, amssymb, amsthm}

% lists
\usepackage{enumerate}

% images
\usepackage{graphicx} % for images
\usepackage{tikz}

% code blocks
% \usepackage{minted, listings} 

% verbatim greek
\usepackage{alphabeta}

\graphicspath{{./assets/images/Module 2}}

\title{02-680 Module 2 \\ \large{Essentials of Mathematics and Statistics}}
 
\author{Aidan Jan}

\date{\today}

\begin{document}
\maketitle

\section*{Number Sets}
\subsection*{Booleans}
A boolean is a data type that takes two values: True and False.
\begin{itemize}
	\item Sometimes we use 1 and 0 instead
	\item Can also be thought of as yes/no, on/off, etc.
	\item can be represented using 1 bit.
\end{itemize}

\subsection*{Integers}
An integer is a number with \textbf{no fractional} part.
\begin{itemize}
	\item We use the symbol $\mathbb{Z}$ to represent the set of \textbf{all possible integers}
	\begin{itemize}
	    \item There is a lot of them (infinitely many!), so unlike booleans we can't just list all integers.
	    \item Can be positive or negative or zero. (As opposed to \textbf{natural numbers}, represented by $\mathbb{N}$, and includes only positive integers)
	    \item The number of bits needed to represent an integer increases with the value.
    \end{itemize}
\end{itemize}

\subsection*{Rational Numbers}
A rational number is one that can be represented by a \textbf{ratio} between two integers.  That is, it can be written in form $\frac{m}{n}$, where $m, n \in \mathbb{Z}$.
\begin{itemize}
	\item We use the symbol $\mathbb{Q}$ to represent the set of all \textbf{rational numbers}.
	\item Real numbers that are not rational are called \textbf{irrational numbers}.
\end{itemize}

\subsection*{Real Numbers}
A real number is one that can be used to \textbf{measure a continuous one-dimensional quantity} such as a length, duration, or temperature.
\begin{itemize}
	\item Includes all integers, rational numbers, and all numbers "between" rational numbers.
	\item We use the symbol $\mathbb{R}$ to represent the set of \textbf{all real numbers}.
	\item Typically we need more bits to store real-valued numbers
	\begin{itemize}
	    \item proportional to the value and precision
	    \item many representations
    \end{itemize}
\end{itemize}

\subsection*{Complex Numbers}
Complex numbers were invented in order to allow for taking the square root of negative numbers
\begin{itemize}
	\item Each complex number has a \textbf{real} and \textbf{imaginary} part, and is written as $a + bi$, where $a, b \in \mathbb{R}$
    \item We denote the set of all \textbf{complex numbers} as $\mathbb{C}$.
    \item Note that $\mathbb{R}$ is a subset of $\mathbb{C}$, where all values of $b$ are 0.
\end{itemize}

\section*{Set Theory}
Why does set theory matter?\\\\
Suppose we define a property like:
\begin{center}
    "Genes that are significantly upregulated under condition $A$"
\end{center}
Then we collect all the genes that match this property.  This collection is a \textbf{set}.\\\\
We may also define a set by listing its members.  Single-cell transcriptomics may identify:
\begin{center}
    Cell type set = \{T cell, B cell, Macrophage, Fibroblast\}
\end{center}
A \textbf{sample space} is the \textbf{collection} of all possible outcomes of an experiment.
\begin{itemize}
	\item The sample space of an experiment can be thought of as a set, or collection, of different possible outcomes; and each outcome can be thought of as a point, or an element, in the sample space.
\end{itemize}

\subsection*{Venn Diagrams}
Venn diagrams can be used to compare sets.  Each circle represents a set, and members common to multiple sets are included in the overlap of the circles.\\\\
Below is an example using differentially expressed genes (DEGs):
\begin{center} 
	\includegraphics*[width=0.5\textwidth]{M2_1.png} 
\end{center}

\subsection*{What is a Set?}
A \textbf{set} is an \textbf{unordered collection} of objects.  We have already been talking about these in the abstract:
\begin{itemize}
	\item Set of all integers $\mathbb{Z}$
	\item Set of all real numbers $\mathbb{R}$
\end{itemize}
We write a set using braces "\{" and "\}"
\begin{itemize}
	\item use capital letters as names (when applicable)
	\item if we wanted to define the set of all algebraic operators we may use $O = \{+, -, \cdot, \div\}$
	\item or maybe the set of all prime numbers: $P = \{2, 3, 5, 7, \dots\}$
\end{itemize}
We have two standard ways of explicitly defining a set.
\begin{itemize}
	\item First is simple (exhaustive) enumeration:
	\[A = \{\text{"Welcome"}, \text{"to"}, \text{"02-680"}\}\]
    In this case the set contains 3 elements, each of which is a string.
    \item The second way is to define a set by abstraction:
    \[B = \{x^2 | x = 2 \lor x = 3\}\]
    We could say that $4 \in B$, but $16 \notin B$.
    \begin{itemize}
	    \item The definition after the $\vert$ is a statement written in \textbf{propositional logic}.
    \end{itemize}
\end{itemize}

\subsection*{Set Membership}
For a set $S$ and an object $x$, the expression $x \in S$ is true when $x$ is one of the objects in the set $S$.
\begin{itemize}
	\item We read $x \in S$ as "$x$ is an element of $S$", or "$x$ is in $S$".
	\item There is also the notion of non-membership
	\begin{itemize}
	    \item if the set $P$ is the set of all primes, then $4 \notin P$.
	    \item that is, $4$ is not in $P$, or $4$ is not prime.
    \end{itemize}
\end{itemize}

\subsection*{Set Cardinality}
For a set $S$, $|S|$ is the number of distinct elements in $S$.\\\\
Often we want to know how large a set is so we can compare sets.
\begin{itemize}
	\item For the set of all algebraic operators described before $|O| = 4$
	\item For the set of all possible bit values $B$, $|B| = 2$.
\end{itemize}
but sometimes, we can't actually count how many there are.  For example,
\begin{itemize}
	\item $|\mathbb{Z}| = \infty$
	\item but, we still know that $|\mathbb{Z}| < |\mathbb{R}|$, since we know that $\mathbb{Z}$ is a subset of $\mathbb{R}$.
\end{itemize}
In the examples mentioned earlier:
\[A = \{\text{"Welcome"}, \text{"to"}, \text{"02-680"}\}\]
\[B = \{x^2 | x = 2 \lor x = 3\}\]
$|A| = 3$ and $|B| = 2$.\\\\
Note that if we define another set:
\[C = \{(2 + 2), (2 - 2), (2 \cdot 2), (2 \div 2), (2^2)\}\]
the cardinality $|C| = 3$.  It is not $5$ since cardinality only includes unique elements.

\subsection*{Set Construction}
As described earlier, there are two ways to construct a set.  Exhaustive enumeration and set abstraction.
\begin{itemize}
	\item Note that not all sets need to be assigned a name.
	\item Remember that sets are \textbf{unordered}, so the following sets are equivalent.
	\begin{itemize}
	    \item $\{2 + 2, 2 \cdot 2, 2 \div 2, 2 - 2\}$
	    \item $\{0, 1, 4\}$
	    \item $\{4, 0, 1\}$
    \end{itemize}
\end{itemize}

\subsubsection*{Set Abstraction}
Let $U$ be a set of possible elements called the \textbf{universe}.  Let $P(x)$ be a condition, also called a \textbf{predicate}, that
\begin{itemize}
	\item for every element $x \in U$
	\item $P(x)$ is \textbf{true} or \textbf{false}.
\end{itemize}
then we can write $\{x \in U \::\: P(x)\}$
\begin{itemize}
	\item which is all of the objects from $x \in U$
	\item for which $P(x)$ is \textbf{true}.
\end{itemize}
For example:
\begin{itemize}
	\item set of even prime numbers: $\{x \in \mathbb{Z} \geq 1\::\: \text{$x$ is prime and $x$ is even}\}$
	\item set of primes between 10 and 20: $\{y \in \mathbb{Z} \::\: \text{$y$ is prime and $10 \leq y \leq 20$}\}$
	\item set of bits: $\{b \in \mathbb{Z} \::\: b^2 = b\}$
\end{itemize}
For convenience, sometimes we don't use the universe if it's obvious
\begin{itemize}
	\item this is particularly helpful when the predicate determines the universe
	\item In that case we could write something like $\{x \::\: P(x)\}$
\end{itemize}
There is, of course, multiple ways to write these sets as well.
\begin{itemize}
	\item Two digit perfect squares: $\{n \in \mathbb{Z} \::\: \sqrt{n} \in \mathbb{Z} \text{ and } 10 \leq n \leq 99\}$
	\item $\{n^2 \::\: n \in \mathbb{Z} \text{ and } 10 \leq n^2 \leq 99\}$
	\item $\{n^2 \::\: n \in \{4, 5, 6, 7, 8, 9\}\}$ are the same set
\end{itemize}

\subsection*{Special Sets}
There are some sets that we use that are so important we define them on their own, they also usually have \textbf{special symbols}.\\\\
The \textbf{Empty Set}, also called the \textbf{null set}, is a set that contains no elements.
\[\emptyset = \{\}\]
Since there are no elements in $\emptyset$, which we can write as $\forall x \::\: x \notin \emptyset$, or $\nexists x \::\: x \in \emptyset$, $|\emptyset| = 0$.
\begin{itemize}
	\item Any set with \textbf{no elements} can be reduced to $\emptyset$
	\item Can also be defined as $\{x \::\: \text{False}\}$ (similar to an if statement with "false" in the condition)
\end{itemize}
Additionally, we have common sets for groups of numbers.  
\begin{itemize}
	\item $\mathbb{N}$: natural numbers (positive whole numbers)
	\item $\mathbb{Z}$: integers (all whole numbers)
	\item $\mathbb{Q}$: rational numbers (numbers representable as fractions)
	\item $\mathbb{R}$: real numbers (decimal numbers)
	\item $\{0, 1\}$: booleans (no special symbol, represents two states)
	\item $\Sigma$: characters (sometimes, "alphabet")
\end{itemize}
Sometimes, we want to slightly restrict these sets, so sometimes you may see something like $\mathbb{Z}^+$, which is equivalent to $x \in \mathbb{Z} | x > 0$, or $\mathbb{Z}^{\geq 0}$, which is equivalent to $x \in \mathbb{Z}|x \geq 0$.

\subsection*{Operators}
We have four standard operators used to compare sets, referred to as \textbf{set operators}.
\begin{itemize}
	\item \textbf{Union} ($\cup$): The union of two sets $S$ and $T$, denoted $S \cup T$, is the set of all elements in either $S$ or $T$ (or both).
	\[S \cup T = \{x \::\: x \in S \text{ or } x \in T\}\]
    \item \textbf{Intersection} ($\cap$): The intersection of two sets $S$ and $T$, denoted $S \cap T$, is the set of all elements in both $S$ and $T$.
    \[S \cap T = \{x \::\: x \in S \text{ and } x \in T\}\]
    \item \textbf{Set Difference} ($\backslash$): Similar to numerical subtraction, given two sets $S$ and $T$, $S \backslash T$ is the set of elements in $S$ but not $T$.
    \[S \backslash T = \{x | x \in S \land x \notin T\}\]
    \begin{itemize}
	    \item For set difference, order matters. $S \backslash T$ and $T \backslash S$ are different sets.
	    \item Sometimes, set difference is represented using the minus ($-$) symbol instead of backslash ($\backslash$).
    \end{itemize}
    \item \textbf{Complement} ($\bar{\:}$): While we can write something like $\tilde{\:} S$, it is normally written as $\bar{S}$.  This represents the set of elements that are not in $S$.
    \[\bar{S} = \tilde{\:} S := \{x \in U | x \notin S\}\]

\end{itemize}

\subsubsection*{Properties of Union and Intersection}
Instead of writing long unions and intersections over multiple sets, we can use indexed notation, just like we do with summation and products.
\begin{align*}
    \bigcup_{i = 1}^n S_i &= S_1 \cup S_2 \cup \dots \cup S_n\\
    \bigcap_{i = 1}^n S_i &= S_1 \cap S_2 \cap \dots \cap S_n
\end{align*}
Since a union combines two sets, the following is true:
\[\max\{|S|, |T|\} \leq |S \cup T| \leq |S| + |T|\]
Since an intersection only contains elements present in both sets, the following is true:
\[0 \leq |S \cap T| \leq \min\{|S|, |T|\}\]

\subsubsection*{Arithmetics and Sets}
We previously saw the use of sums and products like
\[\sum_{i = 3}^5 2^i = 8 + 16 + 32 = 56\]
but we can do the same using a set.  Define $S = \{3, 4, 5\}$, then we can say:
\[\sum_{x \in S} 2^x = 8 + 16 + 36 = 56\]
We can do the same with product, max, and min:
\[\prod_{x \in S} x \hspace{1cm} \max_{x \in S} x \hspace{1cm} \min_{x \in S} x\]

\subsection*{Comparing Sets}
We often need to know if one set is contained within another.  (notice while similar this is different from saying one is larger than another.)
\begin{itemize}
	\item Subset $A \subseteq B$: $A$ is fully contained within $B$.
	\item Superset $A \supseteq B$: $B$ is fully contained within $A$.
	\item Equality $A = B$: $A$ contains all elements of, but no more than, $B$.
\end{itemize}
\begin{center} 
	\includegraphics*[width=\textwidth]{M2_2.png} 
\end{center}

\subsubsection*{Subset ($\subseteq$)}
We say one set is a \textbf{subset} of another if all of its elements are in both sets.
\[S \subseteq T \Leftrightarrow \forall x \in S \::\: x \in T\]
A set $S$ is a subset of a set $T$, written $S \subseteq T$, if every $x \in S$ is also an element of $T$.
\begin{itemize}
	\item In other words, $S \subseteq T$ is equivalent to $S - T = \{\}$.
\end{itemize}

\subsubsection*{Superset ($\supseteq$)}
If a set contains another set, we can say the original is a \textbf{superset} of the other.
\[S \supseteq T \Leftrightarrow \forall x \in T \::\: x \in S\]

\subsubsection*{Equality ($=$)}
If two sets contain the same elements, they are equal.
\[S = T \Leftrightarrow (\forall x \in S\::\: x \in T) \land (\forall y \in T \::\: y \in S)\]
$S$ is equal to $T$ if and only if every element of $S$ is also in $T$, and every element of $T$ is also in $S$.
\[S = T \Leftrightarrow (S \subseteq T) \land (S \supseteq T)\]

\subsubsection*{Proper Subset and Proper Supersets}
We also have the proper subset ($\subset$) and proper superset ($\supset$) operators, which represent subset and superset (but not equality).
\begin{itemize}
	\item We define a proper subset as
	\[S \subset T \Leftrightarrow (\forall x \in S \::\: x \in T) \land (\exists y \in T \::\notin S)\]
    \begin{itemize}
	    \item A set $S$ is a proper subset of a set $T$, written $S \subset T$, if $S \subseteq T$ and $S \neq T$.  In other words, $S \subset T$ whenever $S \subseteq T$, but $T \nsubseteq S$.
    \end{itemize}
    \item We define a proper superset as
    \[S \supset T \Leftrightarrow (\forall x \in T \::\: x \in S) \land (\exists y \in S \::\notin T)\]
    \begin{itemize}
	    \item $S$ is a proper superset of $T$ if and only if every element of $T$ is in $S$, and there exists at least one element in $S$ that is not in $T$.
    \end{itemize}
\end{itemize}

\subsection*{Power Sets}
The \textbf{power set} represents the collection of all subsets of a given set.  It is a set of sets.
\[P(S) := \{T | T \subseteq S\}\]
The \textbf{power set} of $S$ is denoted by $(S)$.\\\\
For example, the power set of ${0, 1, 2}$:
\[(\{0, 1, 2\}) = \{\emptyset, \{0\}, \{1\}, \{2\}, \{0, 1\}, \{0, 2\}, \{1, 2\}, \{0, 1, 2\}\}\]
Also, a power set of a power set is a set of set of sets\dots

\subsubsection*{Cardinality of a Power Set}
An interesting fact is that the size of the power set of a set is
\[|P(S)| = 2^{|S|}\]
Think of each element of $P(S)$ as a binary number of length $|S|$.  If the $i$-th bit (or position) is 1, then the $i$-ith element from $S$ is in that subset.  Then, each binary number represents a unique subset of $S$.

\subsubsection*{Power Set of the Empty Set}
The cardinality of the power set of the empty set is $1$.  ($2^0 = 1$).  This is because even though the empty set is empty, it still has one subset, which is itself.
\[P(\emptyset) = \{\emptyset\}\]
Note that $\{\emptyset\} \neq \emptyset$.  This is a set containing the empty set, not the empty set itself.  In fact,
\[P({\emptyset}) = \{\emptyset, \{\emptyset\}\}\]

\subsection*{Disjoint Sets}
Two sets $S$ and $T$ are disjoint if there is no $x \in S$ where $x \in T$.  In other words, if $S \cap T = {}$.
\begin{center} 
	\includegraphics*[width=0.4\textwidth]{M2_3.png} \\
    Two disjoint sets $S$ and $T$.
\end{center}
For example, the sets $\{1, 2, 3\}$ and $\{4, 5, 6\}$ are disjoint, since their intersection is $\{\}$.  However, $\{2, 3, 5, 7\}$ and $\{2, 4, 6, 8\}$ are not disjoint, because $2$ is an element of both.

\subsection*{Partitions}
The first interesting use of a set of sets is to form a partition of $S$ into a set of disjoint subsets whose union is precisely $S$.
\begin{itemize}
	\item A partition of a set $S$ is a set $\{A_1, A_2, \dots, A_k\}$ of nonempty sets $A_1, A_2, \dots, A_k$, for some $k \geq 1$, such that 
	\begin{itemize}
	    \item $A_1 \cup A_2 \cup \dots \cup A_k = S$
	    \item for any $i, j$, where $j \neq i$, the sets $A_i$ and $A_j$ are disjoint.
    \end{itemize}
\end{itemize}
For example, consider the set $S = \{1, 2, 3, 4, 5, 6, 7, 8, 9, 10\}$.  Here are some different ways to partition $S$:
\begin{itemize}
	\item $\{\{1, 3, 5, 7, 9\}, \{2, 4, 6, 8, 10\}\}$
	\item $\{\{1, 2, 3, 4, 5, 6, 7, 8, 9\}, \{10\}\}$
	\item $\{\{1, 4, 7, 10\}, \{2, 5, 8\}, \{3, 6, 9\}\}$
	\item $\{\{1\}, \{2\}, \{3\}, \{4\}, \{5\}, \{6\}, \{7\}, \{8\}, \{9\}, \{10\}\}$
	\item $\{\{1, 2, 3, 4, 5, 6, 7, 8, 9, 10\}\}$
\end{itemize}

\section*{Use of Sets}
In machine learning we often label data points.  One way we can do that is using nearest neighbor:
\begin{itemize}
	\item Given a set of labelled data and a new unlabeled point
	\item Use the "closest" point to label the new one.
\end{itemize}
This method creates a \textit{partition} of the space into those that are closest to some label.
\begin{center} 
	\includegraphics*[width=\textwidth]{M2_4.png} 
\end{center}

\subsection*{Jaccard Coefficient}
Jaccard measures the \textbf{correlation} of two sets as a ratio of:
\begin{itemize}
	\item how many elements are in both sets, over
	\item how many elements are in either set.
\end{itemize}
\[J(A, B) = \frac{|A \cap B|}{|A \cup B|} = \frac{|A \cap B|}{|A| + |B| - |A \cap B|}\]

\section*{Functions}
We will say that a function provides a \textbf{mapping} from one set onto another.  Formally we say that a function $f$ that maps from a set $S$ to a set $T$ is written as
\[f \::\: S \mapsto T\]
\subsubsection*{Formal Definition}
Let $S$ and $T$ be sets.  A function $f$ from $S$ to $T$, written $f \::\: S \mapsto T$, assigns to each input value $a \in A$ a \textbf{unique} output value $b \in T$; the unique value $b$ assigned to $a$ is denoted by $f(a)$.\\\\
We sometimes say that $f$ maps $a$ to $f(a)$.  Note that $S$ and $T$ are allowed to be in the same set.\\\\
For example, a function might have inputs and outputs that are both elements of $\mathbb{Z}$

\subsection*{Domain, Codomain, and Range}
When writing $f \::\: S \mapsto T$, we call $S$ the \textbf{domain} set, and $T$ the \textbf{codomain}.  Note that the \textbf{codomain} is slightly different from the range of a function; the \textbf{range} is the subset of $T$ that is reachable from an input in $S$.  Formally the range is
\[y \in T | \exists x \in S \::\: f(x) = y\]
The set of all $y$ in $T$ such that there exists an $x$ in $S$ with $f(x) = y$.
\begin{itemize}
	\item Note that not all possible outputs in the codomain are actually achieved.  There may be an element $b \in T$ for which there's no $a \in S$ with $f(a) = b$.
\end{itemize}

\subsection*{Range/Image}
The range (or image) of a function $f \::\: S \mapsto T$ is the set of all $b \in T$ such that $f(a) = b$ for some $a \in S$.  The range of $f$ is the set
\[\{y \in T \::\: \text{ there exists at least one $x \in S$ such that $f(x) = y$}\}\]

\subsection*{Domain and Codomain}
The \textbf{domain} and \textbf{codomain} of a fuction are its sets of possible inputs and outputs:  for a function $f \::\: S \mapsto T$, the set $S$ is called the \textbf{domain}, and the set $T$ is called the \textbf{codomain} of the function $f$.  Some examples:
\begin{itemize}
	\item Not has domain \{True, False\}, and codomain \{True, False\}.  In this case, range = codomain.
	\item The square function maps $\mathbb{R} \rightarrow \mathbb{R}$.  However, in this case, Range $\subset$ Codomain, since the range does not span negative numbers.
\end{itemize}



\end{document}