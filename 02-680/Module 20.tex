% document formatting
\documentclass[10pt]{article}
\usepackage[utf8]{inputenc}
\usepackage[left=1in,right=1in,top=1in,bottom=1in]{geometry}
\usepackage[T1]{fontenc}
\usepackage{xcolor}

% math symbols, etc.
\usepackage{amsmath, amsfonts, amssymb, amsthm}
\usepackage{dsfont}

% lists
\usepackage{enumerate, enumitem}
\usepackage{tabularx, multirow}

% images
\usepackage{graphicx} % for images
\usepackage{tikz}

% code blocks
% \usepackage{minted, listings} 

% verbatim greek
\usepackage{alphabeta}

\newcommand{\dd}{\text{d}}
\newcommand{\var}{\text{Var}}
\DeclareMathOperator*{\argmax}{arg\,max}

\graphicspath{{./assets/images/Module 20}}

\title{02-680 Module 20 \\ \large{Essentials of Mathematics and Statistics}}
 
\author{Aidan Jan}

\date{\today}

\begin{document}
\maketitle

\section*{Hypothesis Testing}
Hypothesis Testing is
\begin{itemize}
	\item A technique to evaluate if a model fit matches our assumptions about the data.
	\item Allows us to assign a numerical value (e.g., $p$-value) to assess this match.
\end{itemize}
There are two main types of hypothesis:
\begin{itemize}
	\item \textbf{Null Hypothesis ($H_0$)}: The assumption that there is no effect or no difference.
	\item \textbf{Alternate Hypothesis ($H_1$)}: The assumption that there is an effect or a difference.
\end{itemize}

\subsection*{Hypothesis Test Outcome}
The test helps us decide to:
\begin{itemize}
	\item Reject $H_0$ \textrightarrow Evidence supports $H_1$.
	\item Retain $H_0$ \textrightarrow Not enough evidence to support $H_1$.
\end{itemize}
Example context: Testing whether a drug impacts cholesterol:
\begin{itemize}
	\item $H_0$: Cholesterol stays the same.  (No effect.)
	\item $H_1$: Cholesterol level changes.
\end{itemize}


\subsection*{Errors}
\begin{itemize}
	\item Type I Error (False Positive): Rejecting the null hypothesis when $H_0$ is actually true.
	\item Type II Error (False Negative): Retaining the null hypothesis when $H_0$ is actually false.
\end{itemize}
\begin{center}
\begin{tabular}{c|cc}
 & \multicolumn{2}{c}{Hypothesis Test Result} \\
Truth & Retain $H_0$ & Reject $H_0$ \\ \hline
$H_0$ & Correct & Type I Error \\
$H_1$ & Type II Error & Correct
\end{tabular}
\end{center}
We say the Type I Errpr Rate is $p(\text{reject } H_0$ | $H_0 \text{ is true})$, and Type II Error Rate is $p(\text{retain } H_0, H_1 \text{ is true})$, and Statistical Power is $1 - \text{Type II Error Rate}$.  The last point means that the higher power tests have a stronger ability to detect signals for $H_1$.

\subsubsection*{Example}
Let's look at it visually, first for what we call a two-sided test, that is
\[H_0 \::\: \mu = x \text{ and } H_1 \::\: \mu \neq x\]
\begin{center} 
	\includegraphics*[width=0.8\textwidth]{M20_1.png} 
\end{center}
In the figure above, when we pick some boundary around our desired $x$ (the green lines) we will have some probability of Type I Error (blue shaded regions) and Type II Error (red shaded region).


\end{document}