% document formatting
\documentclass[10pt]{article}
\usepackage[utf8]{inputenc}
\usepackage[left=1in,right=1in,top=1in,bottom=1in]{geometry}
\usepackage[T1]{fontenc}
\usepackage{xcolor}

% math symbols, etc.
\usepackage{amsmath, amsfonts, amssymb, amsthm}

% lists
\usepackage{enumerate}
\usepackage{tabularx}

% images
\usepackage{graphicx} % for images
\usepackage{tikz}

% code blocks
% \usepackage{minted, listings} 

% verbatim greek
\usepackage{alphabeta}

\graphicspath{{./assets/images/Module 3}}

\title{02-680 Module 3 \\ \large{Essentials of Mathematics and Statistics}}
 
\author{Aidan Jan}

\date{\today}

\begin{document}
\maketitle

\section*{Ordered Collections}
Recall that a set is a well-defined collection of distinct objects.
\begin{itemize}
	\item No duplicates
	\item No order
	\item Suitable for representing "which" elements are in a group.
\end{itemize}
However, in many contexts, order matters.  For example, a set like \{latitude, longitude\} cannot distinguish between $(31.2, 121.4)$ and $(121.4, 31.2)$.  These are different locations!

\subsection*{Sequences}
A sequence, also known as a list or tuple, is an \textbf{ordered collection} of objects.  (typically called components or entries)
\begin{itemize}
	\item When the number of objects in the collection is 2, 3, 4, or $n$
	\item the sequence is called an (ordered pair), triple, quadruple, or $n$-tuple.
	\item In some conventions they may be written using \textbf{angle brackets} $\langle \rangle$, but parenthesis also work.
\end{itemize}
Tuples may be used for representing positions, color, etc.

\subsection*{Tuples}
When these have small cardinality (length) we can use terms like
\begin{itemize}
	\item Ordered pair
	\item Triple
	\item Quadruple
	\item or more generically an "$n$-tuple"
\end{itemize}

\subsection*{Cartesian Product}
A very useful way to \textbf{construct} set of tuples is using the \textbf{cartesian product} operator, which in essence creates all possible pairs of elements from two sets.
\[S \times T = \{\langle x, y \rangle | x \in S \land y \in T\}\]
As an example, let's remember the first two sets from our examples before:
\begin{align*}
    A &= \{\text{"Welcome"}, \text{"to"}, \text{"02-680"}\}\\
    B &= \{x^2 | x = 2 \lor x = 3\} = \{4, 9\}
\end{align*}
In this case, the \textbf{cartesian product} is
\[A \times B = \{\langle \text{"Welcome"}, 4\rangle, \langle \text{"to"}, 4\rangle, \langle \text{"02-680"}, 4\rangle, \langle \text{"Welcome"}, 9\rangle, \langle \text{"to"}, 9\rangle, \langle \text{"02-680"}, 9\rangle\} \]
It doesn't have to be different sets in the cartesian product though, we can have the product with a set and itself.  In fact, this is performed so often it has its own notation:
\[B \times B = B^2 = \{\langle 4, 4 \rangle, \langle 4, 9 \rangle, \langle 9, 4 \rangle, \langle 9, 9 \rangle\}\]
This notation also generalizes, so $S^3 = S \times S \times S$, etc.
\begin{itemize}
	\item Note that $S^2$ will contain all the tuples of length $2$, $S^3$ will contain all the tuples of length $3$, etc.
	\item If a tuple is raised to the zeroth power, it results in an empty tuple.
\end{itemize}

\subsection*{Sets of Tuples with Varying Lengths}
Sometimes we want to have a set of tuples of different lengths.  Remember sets don't need to be over objects of the same type.\\\\
Let $B = \{4, 9\}$.  In this case, $B^2$ represents all the 2-tuples over $B$, and $B^3$ represents all the 3-tuples over $B$.
\begin{itemize}
	\item Therefore, the union $B^2 \cup B^3$ will contain both, even though the tuples are not the same length.
\end{itemize}

\subsubsection*{Kleene Star}
If we wanted to enumerate all binary numbers up to 8 digits (while omitting leading 0's), how do we express this set?
\begin{itemize}
	\item We can use a cartesian product and a union to define the set:
	\[\{1\} \times \bigcup_{i = 1}^7 \{0, 1\}^i\]
\end{itemize}
The $\{1\}$ ensures the first digit is 1.  The union represents the other digits.  However, sometimes we want the set of all tuples of \textbf{any length}, then we use the \textbf{Klenne Star}, denoted by $*$ to generate all finite-length tuples formed from elements of a set.
\begin{itemize}
	\item For some set $S$, we define $S^* = \bigcup_{i = 0}^\infty S^i$.
	\item $S^0 = \langle \rangle$ (the empty tuple) for any $S$, and in the case of $\Sigma^0$, we often call it the \textbf{empty string}.
\end{itemize}

\subsection*{Strings as Tuples Over an Alphabet}
In certain contexts, sequences of elements from the same set are called \textbf{strings}.  For a set $\Sigma$ called an alphabet, a string over $\Sigma$ is an element of $\Sigma^n$, for some nonnegative integer $n$.\\\\
In other words, a string is any element of $\bigcup_{n \in Z \geq 0} \Sigma^n$.  The length of a string $x \in \Sigma^n$ is $n$.
\begin{itemize}
	\item For example, the set of all 5-letter English words is a subset of:
	\[\{A, B, \dots, Z\}^5\]
\end{itemize}

\subsubsection*{Notation Conventions for Strings}
When writing strings, we usually omit angle brackets and commas (unlike regular tuples.)


\subsubsection*{Empty String}
Strings are just sets of characters in an alphabet.  They can have any non-negative length, including zero.  For an alphabet $\Sigma$:
\[\Sigma^0 \{\epsilon\}\]
Here, $\epsilon$ is the empty string:
\begin{itemize}
	\item It contains zero elements
	\item It is unique (only one such string exists)
\end{itemize}
\begin{center}
    \begin{tabularx}{\textwidth}{XXX}
    \textbf{String} & \textbf{Belongs to} & \textbf{Explanation} \\
    ABRACADABRA & $\{A, B, \dots, Z\}^{11}$ & A string of 11 capital letters \\
    11010011 & $\{0, 1\}^8$ & A binary string of length 8
    \end{tabularx}
\end{center}
\end{document}
  