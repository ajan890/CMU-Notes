% document formatting
\documentclass[10pt]{article}
\usepackage[utf8]{inputenc}
\usepackage[left=1in,right=1in,top=1in,bottom=1in]{geometry}
\usepackage[T1]{fontenc}
\usepackage{xcolor}

% math symbols, etc.
\usepackage{amsmath, amsfonts, amssymb, amsthm}
\usepackage{dsfont}

% lists
\usepackage{enumerate, enumitem}
\usepackage{tabularx}

% images
\usepackage{graphicx} % for images
\usepackage{tikz}

% code blocks
% \usepackage{minted, listings} 

% verbatim greek
\usepackage{alphabeta}

\graphicspath{{./assets/images/Module 11}}

\title{02-680 Module 11 \\ \large{Essentials of Mathematics and Statistics}}
 
\author{Aidan Jan}

\date{\today}

\begin{document}
\maketitle

\section*{Calculus Review}
\subsection*{Derivatives}
For some function $f(x)$ where $x$ is a scalar, 
\begin{itemize}
	\item the derivative $\frac{\dd f}{\dd x}$ is the change in the value of $f$ as you increase/decrease $x$.
	\item formally, $\frac{\dd f}{\dd x} = \lim_{h \rightarrow 0} \frac{f(x + h) - f(x)}{h}$
	\item Know basic derivatives.  (e.g., polynomials, trig functions, inverse trig functions, exponentials, logarithms)
	\item Know basic derivative rules.  (e.g., chain rule, product rule, quotient rule)
\end{itemize}

\subsection*{Integrals}
For some function $f'(x)$ where $x$ is a scalar, the integral $\int f'(x)$ can be thought of as the inverse of differentiation.\\\\
For example:
\begin{align*}
    f'(x) &= nx^{n - 1} \\
    \int nx^{n - 1} \dd x &= n \int x^{n - 1} \dd x \\
    &= n \cdot \frac{x^n}{n} + C
    &= x^n + C
\end{align*}
Know the following:
\begin{itemize}
	\item Rules: constant multiple rule, polynomial rule, exponent rules, log rule, sum rule.
\end{itemize}

\subsection*{Application: Gradient Based Optimization}
\textbf{Optimization} is the task of either minimizing or maximizing some function.\\\\
For some $f(x)$, find the $x$ that maximizes $f(x)$.  Using mathematics:
\[x^* = \argmax_{\forall x} f(x)\]
Note that above we use maximization, but this would be equivalent to minimizing some function $g(x) = -f(x)$.

\subsection*{Gradient Descent}
Remember the derivative of a function is the rate of change, or slope; thus it can be used to tell us how to change $x$ in order to have a maximizing impact on $f(x)$.
\[\frac{\dd f}{\dd x} \approx \frac{f(x + \epsilon) - f(x)}{\epsilon} \rightarrow f(x + \epsilon) \sim f(x) + \epsilon f'(x) \rightarrow f(x + \epsilon) - f(x) \sim \epsilon f'(x)\]
When $f'(x) = 0$ we call this a \textbf{critical}, or \textbf{stationary}, point.

\end{document}