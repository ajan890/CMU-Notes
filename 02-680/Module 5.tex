% document formatting
\documentclass[10pt]{article}
\usepackage[utf8]{inputenc}
\usepackage[left=1in,right=1in,top=1in,bottom=1in]{geometry}
\usepackage[T1]{fontenc}
\usepackage{xcolor}

% math symbols, etc.
\usepackage{amsmath, amsfonts, amssymb, amsthm}
\usepackage{dsfont}

% lists
\usepackage{enumerate, enumitem}
\usepackage{tabularx}

% images
\usepackage{graphicx} % for images
\usepackage{tikz}

% code blocks
% \usepackage{minted, listings} 

% verbatim greek
\usepackage{alphabeta}

\graphicspath{{./assets/images/Module 5}}

\title{02-680 Module 5 \\ \large{Essentials of Mathematics and Statistics}}
 
\author{Aidan Jan}

\date{\today}

\begin{document}
\maketitle

\section*{Vectors}
A vector is a quantity with both magnitude and direction.  It is either written as $\vec{v}$, or $V$.
\begin{itemize}
	\item An $n$-dimensional vector $x$ is defined as an element of the \textbf{Cartesian product} of $n$ copies of the real numbers:
	\[x \in \mathbb{R}^n\]
    \item This means that $x = (x_1, x_2, \dots, x_n)$, and if we want to reference the $i$-th element of $x$ we will write $x_i$.
    \item (or sometimes $x[i]$.  This is true of tuples as well)
\end{itemize}
For example, $\langle 0, 1 \rangle$, $\langle 1, 0 \rangle$, and $\langle \frac{1}{2}, \frac{1}{2} \rangle$ are all vectors of length $2$ where elements are from $\mathbb{Q}$ (or we would probably say $\mathbb{R}$).
\begin{itemize}
	\item For the vector $x = \langle \frac{1}{2}, \frac{3}{2} \rangle$, we have $x_1 = \frac{1}{2}$, and $x_2 = \frac{3}{2}$
\end{itemize}
Vectors are sometimes contrasted with scalars, which are just numbers:
\begin{itemize}
	\item That is, a \textbf{scalar} is an element of $\mathbb{R}$
\end{itemize}
Vectors are also sometimes written in parenthesis, so we may see an $n$-vector $x$ written as $x = (x_1, x_2, \dots, x_n)$

\subsection*{Vector Arithmetic}
\subsubsection*{Sum of Vectors}
The \textbf{sum} of two vectors $x, y \in \mathbb{R}^n$, written $x + y$, is a vector $z \in \mathbb{R}^n$
\begin{itemize}
	\item where for every index $i \in \{1, 2, \dots, n\}$ we have $z_i = x_i + y_i$
	\item Note: the sum of two vectors with different sizes is meaningless
\end{itemize}
Examples:
\begin{itemize}
	\item $\langle 3, 4\rangle + \langle 2, 1 \rangle = \langle 5, 5 \rangle$
	\item $\langle 1.1, 2.2, 3.3 \rangle + \langle 4.4, 5.5, 6.6 \rangle = \langle 5.5, 7.7, 9.9 \rangle$
	\item $\langle 0, 1, 2 \rangle + \langle 3, 4 \rangle$ does not make sense!
	\item Vector addition operations are element-wise.
\end{itemize}
For $n = \{1, 2, 3\}$, we can visualize vectors.
\begin{itemize}
	\item We usually draw them as arrows.
	\item This can help us interpret operations like addition.
\end{itemize}
\begin{center} 
	\includegraphics*[width=\textwidth]{M5_1.png} 
\end{center}
Note that the order does not matter since addition is commutative.

\subsubsection*{Scalar Product}
Given a vector $x \in \mathbb{R}^n$, and a real number $a \in \mathbb{R}$ the \textbf{scalar product}
\begin{itemize}
	\item is a vector $z \in \mathbb{R}^n$,
	\item written $z = ax$,
	\item where for every index $i \in \{1, 2, \dots, n\}$ we have $z_i = ax_i$.
\end{itemize}
Examples:
\begin{itemize}
	\item $2 \langle 3, 4 \rangle = \langle 6, 8 \rangle$
	\item $4.4 \langle 1.1, 2.2, 3.3 \rangle = \langle 4.84, 9.68, 14.52 \rangle$
\end{itemize}


\end{document}