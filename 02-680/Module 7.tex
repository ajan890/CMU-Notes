% document formatting
\documentclass[10pt]{article}
\usepackage[utf8]{inputenc}
\usepackage[left=1in,right=1in,top=1in,bottom=1in]{geometry}
\usepackage[T1]{fontenc}
\usepackage{xcolor}

% math symbols, etc.
\usepackage{amsmath, amsfonts, amssymb, amsthm}
\usepackage{dsfont}

% lists
\usepackage{enumerate, enumitem}
\usepackage{tabularx}

% images
\usepackage{graphicx} % for images
\usepackage{tikz}

% code blocks
% \usepackage{minted, listings} 

% verbatim greek
\usepackage{alphabeta}

\graphicspath{{./assets/images/Module 7}}

\title{02-680 Module 7 \\ \large{Essentials of Mathematics and Statistics}}
 
\author{Aidan Jan}

\date{\today}

\begin{document}
\maketitle

\section*{Linear Systems of Equations}
We can define a \textbf{linear system} of $n$ linear equations on $m$ variables as follows:
\[\begin{matrix}
C_{11} x_1 &+& C_{12} x_2 &+& \cdots &+& C_{1m}x_m &=& b_1 \\
C_{21} x_1 &+& C_{22} x_2 &+& \cdots &+& C_{2m}x_m &=& b_2 \\
\vdots & & \vdots & & \ddots & & \vdots & & \vdots \\
C_{n1} x_1 &+& C_{n2} x_2 &+& \cdots &+& C_{nm}x_m &=& b_n
\end{matrix}\]
As an example:
\[\begin{cases} 3z_1 + 2z_2 = -1 \\ z_1 - 5z_2 = 3 \end{cases}\]
\[\begin{bmatrix} 3 & 2 \\ 1 & -5 \end{bmatrix} \begin{bmatrix} z_1 \\ z_2 \end{bmatrix} = \begin{bmatrix} -1 \\ 3 \end{bmatrix}\]
If we want to find $x$ (or in our example $z$), we can say it is
\[x = C^{-1}b\]
where $C^{-1}$ is the inverse matrix of $C$.  It turns out this matrix may not always exist.
\begin{itemize}
	\item The inverse is the matrix such that $CC^{-1} = C^{-1}C = I_n$ (thus the first condition to the inverse existing is if $C$ is square).
\end{itemize}

\end{document}