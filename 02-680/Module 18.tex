% document formatting
\documentclass[10pt]{article}
\usepackage[utf8]{inputenc}
\usepackage[left=1in,right=1in,top=1in,bottom=1in]{geometry}
\usepackage[T1]{fontenc}
\usepackage{xcolor}

% math symbols, etc.
\usepackage{amsmath, amsfonts, amssymb, amsthm}
\usepackage{dsfont}

% lists
\usepackage{enumerate, enumitem}
\usepackage{tabularx, multirow}

% images
\usepackage{graphicx} % for images
\usepackage{tikz}

% code blocks
% \usepackage{minted, listings} 

% verbatim greek
\usepackage{alphabeta}

\newcommand{\dd}{\text{d}}
\newcommand{\var}{\text{Var}}


\graphicspath{{./assets/images/Module 18}}

\title{02-680 Module 18 \\ \large{Essentials of Mathematics and Statistics}}
 
\author{Aidan Jan}

\date{\today}

\begin{document}
\maketitle

\section*{Parameter Estimation and Maximum Likelihood Estimation}
Consider a dataset $X_1, X_2, \dots, X_n$ of independent and identically distributed random varialbes from the same unknown distribution.  That is, for each $X_{i}$, the underlying distributions have the same $\mu$ and $\sigma$.\\\\
Let $\bar{X_n}$ be the average of the actual value of the observations:
\[\bar{X_n} = \frac{X_1 + X_2 + \cdots + X_n}{n} = \frac{1}{n} \sum_{i = 1}^n X_i\]
Note this is different from the expected value of the underlying distribution $\mu$.  Note also that $\bar{X_n}$ is also a random variable in an of itself.

\subsection*{Statistical Inference}
We will use statistical inference or \textbf{learning} to try to make the models match the observations we've made about the world.

\subsubsection*{Example: Coin Flip Simulator}
Consider a coin flip simulator which usually outputs ``heads'' or ``tails'', but once in every 1000 flips, it outputs a nonsense number.  The simulator acts as a data generator, producing observations we can see and measure.

\subsection*{Model vs. Reality}
On the other side, we have a set of models - mathematical representations of how we think the data might have been generated.  Each model has its own set of parameters, e.g., the probability of heads in a biased coin.

\subsection*{What is Statistical Inference?}
\textbf{Statistical Inference} is the task of:
\begin{itemize}
	\item using the data (observations)
	\item to estimate or ``learn'' the parameters of the model
	\item so that the model better matches the observed data
\end{itemize}
It's about bridging the gap between theory (models) and reality (data).



\end{document}