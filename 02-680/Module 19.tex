% document formatting
\documentclass[10pt]{article}
\usepackage[utf8]{inputenc}
\usepackage[left=1in,right=1in,top=1in,bottom=1in]{geometry}
\usepackage[T1]{fontenc}
\usepackage{xcolor}

% math symbols, etc.
\usepackage{amsmath, amsfonts, amssymb, amsthm}
\usepackage{dsfont}

% lists
\usepackage{enumerate, enumitem}
\usepackage{tabularx, multirow}

% images
\usepackage{graphicx} % for images
\usepackage{tikz}

% code blocks
% \usepackage{minted, listings} 

% verbatim greek
\usepackage{alphabeta}

\newcommand{\dd}{\text{d}}
\newcommand{\var}{\text{Var}}


\graphicspath{{./assets/images/Module 19}}

\title{02-680 Module 19 \\ \large{Essentials of Mathematics and Statistics}}
 
\author{Aidan Jan}

\date{\today}

\begin{document}
\maketitle

\section*{Maximum a Posteriori Estimation}
\subsection*{Frequentist vs. Bayesian Schools}
The \textbf{Frequentist School} assumes that $H$ is fixed and not random.  It uses likelihood only:
\[L(H | D) = p(D | H)\]
Probabilities reflect long-run frequencies.  Relies only on observed data.\\\\
The \textbf{Bayesian School} takes $H$ to be a hypothesis (parameter) and $D$ some data.  Different people will have different a priori beliefs, but we would still like to make useful inferences from the data.\\\\
When $p(H)$ is known, there is no disagreement, we will all just follow Bayes' Rule as written.
\begin{center} 
	\includegraphics*[width=\textwidth]{M19_1.png} 
\end{center}
In practice, there is no universally-accepted prior.  The main n philosophical difference concerns the \textbf{meaning of probability}.
\begin{itemize}
	\item The Frequentist school represents the idea that \textit{probabilities represent long-term frequencies of repeatable random experiments}
	\begin{itemize}
	    \item \textbf{Objective interpretation}
	    \item Example: `A coin has 0.5 probability of tails' means that the relative frequency of tails goes to 0.5 as the number of flips goes to infinity.
    \end{itemize}
    \item The Bayesian school represents the idea that \textit{probability is an abstract concept that measures a state of knowledge or a degree of belief in a given population}
    \begin{itemize}
	    \item \textbf{Subjective interpretation}
	    \item Example: `A coin has 0.5 probability of tails' means you "believe" that you will get tails $50\%$ of the time.
	    \item That is, they consider a range of values each with its own probability of being true.
    \end{itemize}
\end{itemize}

\end{document}