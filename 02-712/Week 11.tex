% document formatting
\documentclass[10pt]{article}
\usepackage[utf8]{inputenc}
\usepackage[left=1in,right=1in,top=1in,bottom=1in]{geometry}
\usepackage[T1]{fontenc}
\usepackage{xcolor}

% math symbols, etc.
\usepackage{amsmath, amsfonts, amssymb, amsthm}

% lists
\usepackage{enumerate}

% images
\usepackage{graphicx} % for images
\usepackage{tikz}

% code blocks
\usepackage{minted, listings} 

% verbatim greek
\usepackage{alphabeta}

\newcommand{\dd}{\text{d}}
\newcommand{\pr}{\text{Pr}}

\graphicspath{{./assets/images/Week 11}}

\title{02-712 Week 11 \\ \large{Biological Modeling and Simulation}}
 
\author{Aidan Jan}

\date{\today}

\begin{document}
\maketitle

\section*{Stochastic Processes}
Suppose we have a state space and transition probabilities.  Let $S = \{0, 1\}$  Let $p$ represent the probability of transitioning from $0 \rightarrow 1$, and $q$ represent the probability of transitioning from $1 \rightarrow 0$.\\\\
Let us define the starting probabilities
\begin{itemize}
	\item $P(x_0 = 0) = \pi_0(0)$
	\item $P(x_0 = 1) = \pi_1(0)$
\end{itemize}
Now, suppose we want to calculate the rate of change of the system:
\[\frac{\dd x}{\dd t} = rx\]
We want to find $r$.  How do we do this?  We first write the equation to find the probability of some given state, $t$ that the state is zero.
\[P(x_t = 0) = (1 - p) P(x_{t - 1} = 0) + q P(x_{t - 1} = 0)\]
Simplifying,
\begin{align*}
    &= (1 - p) P(x_{t - 1} = 0) + q(1 - P(x_{t - 1} = 0)) \\
    &= (1 - p - q) P(x_{t - 1} = 0)
\end{align*}
We can convert this recursive equation to the following:
\[P(x_t = 0) = (1 - p - q)^t \pi_0(0) + q\left(\sum_{j = 0}^{t - 1} (1 - p - q)^j\right)\]
We can simplify this:
\begin{align*}
    &= \frac{q}{p + q} + (1 - p - q)^t \left(\pi_0(\infty) - \frac{q}{p + q}\right) \\
    &= \frac{q}{p + q}
\end{align*}
The right term goes to zero since at large $t$ (e.g., the model has been running for a long time), the $(1 - p - q)^t$ term approaches zero.\\\\
We know that $P(x_t = 1) = 1 - P(x_t = 0)$, so therefore, 
\[P(x_t = 1) = \frac{p}{p + q}\]
We can generate the \textbf{stationary distribution} from these results, which describes the stochastic distribution after a long time of running the model.
\[\left\{\frac{q}{p + q}, \frac{p}{p + q}\right\}\]


\subsection*{The Pure Birth Model}
In this model, we have a birth rate and a starting population, and that's it.
\begin{itemize}
	\item No death
	\item Birth rate is $b$ for all bacterial
	\item Birth rate is independent across individuals
\end{itemize}
Let $m(t)$ be the function that represents birth.  We are given the rate of birth,
\[\frac{\dd m}{\dd t} = bm(t)\]
to start.  Therefore, $m(t) = e^{bt}$.  Suppose we want to find the population at some state, when the population is $n$.  At each time step, since this model is stochastic, there are two possible outcomes:
\begin{itemize}
	\item Either one birth can occur (and therefore, $x_t = m x_{t - \Delta t} = m - 1$)
	\item No births can occur ($x_t = mx_{t - \Delta t} = m - 1$)
\end{itemize}


We can write
\[P_m(t + \Delta t) = P(x_{t + \Delta t} = m) = P_{m - 1}(t) \cdot b \Delta t(m - 1) + P_m(t) (1 - b\Delta t m)\]
We know from the rate of birth, $\frac{\dd P_m(t)}{\dd t} = \frac{P_m(t + \Delta t) - P_m(t)}{\Delta t} =b(m - 1) P_{m - 1}(t) - bm P_m(t)$


\end{document}