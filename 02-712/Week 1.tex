% document formatting
\documentclass[10pt]{article}
\usepackage[utf8]{inputenc}
\usepackage[left=1in,right=1in,top=1in,bottom=1in]{geometry}
\usepackage[T1]{fontenc}
\usepackage{xcolor}

% math symbols, etc.
\usepackage{amsmath, amsfonts, amssymb, amsthm}

% lists
\usepackage{enumerate}

% images
\usepackage{graphicx} % for images
\usepackage{tikz}

% code blocks
\usepackage{minted, listings} 

% verbatim greek
\usepackage{alphabeta}

\graphicspath{{./assets/images/Week 1}}

\title{02-712 Week 1 \\ \large{Biological Modeling and Simulation}}
 
\author{Aidan Jan}

\date{\today}

\begin{document}
\maketitle

\section*{Course Topics}
\subsection*{Models for Optimization}
\begin{itemize}
	\item Ex. Sequence Similarity: Given a DNA or a protein, find other DNA or proteins that are similar.  
	\begin{itemize}
	    \item We can model this problem as a sequence alignment problem, which can be optimized.
	    \item We can also model this problem as an alignment free problem, for example, dividing the DNA or protein into K-mers and comparing them.
    \end{itemize}
	\item This class focuses not on how to align the DNA or protein, but instead focuses on why we would use one representation of the problem over another.
	\begin{itemize}
	    \item Why do we pick one representation over another?
	    \item What are the parameters for the problem?
    \end{itemize}
\end{itemize}

\subsection*{Simulation and Sampling}
Suppose $A$, $B$, and $C$ are chemicals, such that $A + B \rightarrow C$, with rate $k$.
\begin{itemize}
	\item How can you simulate the concentrations of $A$, $B$, or $C$ over time?
	\item Maybe use a graph, or perhaps differential equations (mass action model).
	\item Or, we can do a discrete system.  Start with some amount of $A$, and $B$, and roll a dice, and if it is greater than a certain number, make a $C$.
	\item Why would we pick one representation of the problem over another?
\end{itemize}

\subsection*{Model Inference and AI Modeling}
Suppose we have chemicals $A$, $B$, and $C$, and $A + B \rightleftharpoons C$.  Forward reaction with rate $k_1$, and backwards reaction with rate $k_2$.  We did the reaction and recorded the concentrations over time (discrete), and plotted them on a graph.
\begin{itemize}
	\item How do we pick parameters of the model to match the data?
	\item Maybe find the values of $k_1$ and $k_2$ that best fit the system?
	\item How do we even know the correct equation is $A + B \rightleftharpoons C$?  It could be $A + 2B \rightleftharpoons C$, etc.  (Structural Inference)
\end{itemize}

\subsection*{Model Building}
Consider the evolutionary tree building problem.  You have species human, gorilla, cow, chicken, tuna, and fly.  How do we model this, or how do we design a program to model this?\\\\
Questions to consider:
\begin{enumerate}
	\item How are we going to answer this?
	\item What information do we have to work with?  (e.g., phenotypes, genetic information, etc.)
	\item Which properties matter a lot, and which do not?
	\item What assumptions can we make?
\end{enumerate}





\end{document}