% document formatting
\documentclass[10pt]{article}
\usepackage[utf8]{inputenc}
\usepackage[left=1in,right=1in,top=1in,bottom=1in]{geometry}
\usepackage[T1]{fontenc}
\usepackage{xcolor}

% math symbols, etc.
\usepackage{amsmath, amsfonts, amssymb, amsthm}

% lists
\usepackage{enumerate}

% images
\usepackage{graphicx} % for images
\usepackage{tikz}

% code blocks
\usepackage{minted, listings} 

% verbatim greek
\usepackage{alphabeta}

\newcommand{\dd}{\text{d}}

\graphicspath{{./assets/images/Week 5}}

\title{02-712 Week 5 \\ \large{Biological Modeling and Simulation}}
 
\author{Aidan Jan}

\date{\today}

\begin{document}
\maketitle

\section*{Deterministic Systems - Linear Model with Multiple Variables}
Depending on the type of problem, if it is possible to ignore probabilities, you should always try to do so
\begin{itemize}
	\item That is, abstract out the randomness of the biological system.
\end{itemize}

\subsection*{Example 1}
Condsider a bacterial growth system.  ($r$ is the rate.)
\begin{align*}
    \frac{\dd m}{\dd t} &= r \cdot m
\end{align*}
In this case, the solution is $m(t) = m(0) \cdot e^{rt}$.  If you can predict how the system starts, then you can predict how the system ends.  If $r > 0$, then the number of bacteria increase.  If $r < 0$, then the number of bacteria decrease.  Etc.\\\\
Suppose we want to find when the system is at equilibrium.  The system \textit{is} at equilibrium if $r = 0$, but we can't just change the $r$ since that is defined by how fast the bacteria divide.  Instead now, our only choice is to make $m(t) = 0$.\\\\
In general, for most systems we cannot just solve the differential equation, and we must instead make an estimate.

\subsubsection*{Phase Diagrams}
We can use phase diagrams to determine whether a equilibrium is stable.  A phase diagram (for now in 1D) is a number line, in which vectors are drawn in the direction of the derivative.  
\begin{itemize}
	\item In our example, if $r < 0$, then following the derivative formula, the number of bacteria would tend to zero.  (There are no negative bacteria!)  Therefore, equilibrium is stable.
	\item If $r > 0$, then following the derivative formula, the number of bacteria will explode and not stop at any number.  Therefore the equilibrium is not stable.
\end{itemize}
In general, an equilibrium is stable if when you keep applying the rule of change, the end result tends to some number.  An equilibrium is unstable if end result tends to an infinity.

\subsection*{Example 2}
Now, consider the system:
\begin{align*}
    \frac{\dd m_1}{\dd t} &= r_1 m_1(t)\\
    \frac{\dd m_2}{\dd t} &= r_2 m_2(t)
\end{align*}
This system has two variables (we are now on a 2D plane instead of a number line).
\begin{itemize}
    \item We can solve this system too.  (Since the two equations are independent from each other, it is same as the first system twice.)
    \begin{align*}
        m_1(t) &= m_1(0) e^{r_1 t}\\
        m_2(t) &= m_2(0) e^{r_2 t}
    \end{align*}
	\item By looking at this system, we know that the equilibrium is at $(0, 0)$.
	\item Also, since the two equations are independent, some things can be inferred about the biological system, such as, the two bacteria must not be competing for resources, space, etc.
	\item The phase diagram for this system is now a vector field on a 2D plane.
\end{itemize}
This problem is modeled with an \textbf{affine} model.  Essentially, describing the model using vectors and matrices.  We can rewrite this system as
\[\frac{\dd \vec{m}}{\dd t} = \begin{bmatrix} \frac{\dd m_1}{\dd t} \\ \frac{\dd m_2}{\dd t}\end{bmatrix} = M \begin{bmatrix} m_1 \\ m_2 \end{bmatrix}, \text{ where } M = \begin{bmatrix} r_1 & 0 \\ 0 & r_2 \end{bmatrix}\]
Therefore,
\[\frac{\dd \vec{m}}{\dd t} = M \cdot \vec{m} + \vec{C}\]

\subsection*{Example 3}
Consider now, the two populations of bacteria depend on each others' rates.
\begin{align*}
    \frac{\dd m_1}{\dd t} &= a m_1(t) + b m_2(t) \\
    \frac{\dd m_2}{\dd t} &= c m_1(t) + d m_2(t)
\end{align*}
Therefore, 
\[\frac{\dd \vec{m}}{\dd t} = M \vec{m}, \text{ where } M = \begin{bmatrix} a & b \\ c & d \end{bmatrix}\]
This model has only one equilibrium, provided that $\det(M) \neq 0$.
\begin{itemize}
	\item If $\det(M) = 0$, then there are an infinite number of solutions.  (Non-zero determinant means the matrix is invertible.)
\end{itemize}

\subsection*{Example 4}
Metastasis is a process in which cancer cells spread through the body.  Sometimes cancer cells move via the bloodstream and become lodged into capillaries, in which they start new tumors.  We will construct a model for the dynamics of the number of cancer cells lodged in the capillaries of an organ, $C$, and the number of cancer cells that have invaded the organ $I$.  Suppose that cells are lost from the capillaries by dislodgement or death at a per capita rate $\delta_1$ and that they invade the organ $I$ from the capillaries by a per capita rate $\beta$.  Once cells are in the organ $I$, they die at a rate of $\delta_2$, and the cancer cells in $I$ replicate at a per capita rate $\rho$.  All of the parameters are assumed positive.\\\\
First, we can summarize the information:
\begin{itemize}
	\item Parameters: $\delta_1, \delta_2, \beta, \rho$
	\item Variables: $C(t), I(t)$.  (Number of cancer cells in capillaries and organs, respectively)
	\item Assumptions: All parameters are positive.  No randomness.  Things not mentioned here don't matter.
\end{itemize}
Next, we can write the rules of change:
\begin{align*}
    \frac{\dd C}{\dd t} &= - C(t) \cdot (\delta_1 + \beta)\\
    \frac{\dd I}{\dd t} &= I(t) \cdot (\rho - \delta_2) + C(t) \cdot \beta
\end{align*}
Now, we can convert the system of differential equations into the affine form:
\[M = \begin{bmatrix} - \beta - \delta_1 & 0 \\ \beta & -\delta_2 + \rho \end{bmatrix}\]


\end{document}