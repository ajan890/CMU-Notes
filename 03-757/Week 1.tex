% document formatting
\documentclass[10pt]{article}
\usepackage[utf8]{inputenc}
\usepackage[left=1in,right=1in,top=1in,bottom=1in]{geometry}
\usepackage[T1]{fontenc}
\usepackage{xcolor}

% math symbols, etc.
\usepackage{amsmath, amsfonts, amssymb, amsthm}

% lists
\usepackage{enumerate}

% images
\usepackage{graphicx} % for images
\usepackage{tikz}

% code blocks
% \usepackage{minted, listings} 

% verbatim greek
\usepackage{alphabeta}

\graphicspath{{./assets/images/Week 1}}

\title{03-757 Week 1 \\ \large{Synthetic Biology}}
 
\author{Aidan Jan}

\date{\today}

\begin{document}
\maketitle
\section*{What is Synthetic Biology?}
\begin{itemize}
	\item Synthetic chemistry: synthetize chemicals not found in nature
	\item Synthetic biology: create new biological parts / systems that do not exist in nature
	\item National Academies of Science (2013): "combines both scientific [fill]"
\end{itemize}

\subsection*{What is the difference between synthetic biology, genetic engineering, metabolic engineering, biological engineering?}
\begin{itemize}
	\item Engineering approach to cell and molecular biology
	\item For studying cellular systems in addition to technology development
	\item Modular design
\end{itemize}

\section*{Parts, Modules, Computation}
Think about biological parts as computers.  Each biological part does something different, similar to how computers each compute something different.  Each different part is a module, and you can put the modules together in different ways.
Different Levels of biology, and their significance:
\begin{itemize}
	\item Four biological macromolecules (proteins, lipids, carbohydrates, nucleic acids.  Maybe we can add more?)
	\item Central Dogma (DNA to RNA to proteins\dots generally.  Some viruses can duplicate RNA or go from RNA to DNA.)
	\item Genetic Codons (RNA to proteins.  We can use a process called codon expansion to add new DNA bases and amino acids.)
	\item Organelles (Compartments where cells do different things.)
	\item Cells (Cells may be different in structure depending on their function.  e.g., skin vs. neurons.)
	\item Cell Division (Mitosis.  Can we make something that can duplicate itself?  Are viruses life?)
	\item Biological Systems (Can reproduce, but can also evolve.)
\end{itemize}

\subsection*{Biocomputing}
The fundamental question is, can you compute using biological parts?\\\\
Example biological circuit:
\begin{itemize}
	\item Signals A and B are transcription factors for Gene 1 and Gene 2, which each code for a transcription factor.  
    \item Gene 3 requires the products of both Gene 1 and Gene 2 in order to begin transcription.  
    \item Therefore, if transcription factors A and B are present, then the product of Gene 3 is formed.  This is an AND gate.
\end{itemize}




\end{document}