% document formatting
\documentclass[10pt]{article}
\usepackage[utf8]{inputenc}
\usepackage[left=1in,right=1in,top=1in,bottom=1in]{geometry}
\usepackage[T1]{fontenc}
\usepackage{xcolor}

% math symbols, etc.
\usepackage{amsmath, amsfonts, amssymb, amsthm}

% lists
\usepackage{enumerate}

% images
\usepackage{graphicx} % for images
\usepackage{tikz}

% code blocks
% \usepackage{minted, listings} 

% verbatim greek
\usepackage{alphabeta}

\graphicspath{{./assets/images/Week 4}}

\title{02-613 Week 4 \\ \large{Algorithms and Advanced Data Structures}}
 
\author{Aidan Jan}

\date{\today}

\begin{document}
\maketitle

\section*{A-Star Algorithm}
Suppose we have a GPS, and we want to get from Pittsburgh to Philly.  Currently, we have discussed how to do that with Dijkstra's algorithm, however:
\begin{itemize}
	\item This requires us to explore paths to nodes on the whole map!
	\item Slow, and most calculations are unnecessary.
	\item Is there a better way?
\end{itemize}
In our GPS case, we would want to tell the algorithm, somehow, you should search paths going east.
\begin{itemize}
	\item At the same time though, depending on where you are, not every path should go east.  Maybe a highway starts slightly to the west, and it will save time.
	\item How do we decide how far west is allowed?
\end{itemize}
What if we use the straight line distance as an estimate to which node (city) to visit next?
\begin{itemize}
	\item In addition to the \texttt{d[]} = $\infty$ for the start of Dijkstra's, we also initialize every city with a \texttt{h[]} = (straight line distance to destination).
	\item The \texttt{h[]} is called a \textit{heuristic}.
\end{itemize}
\begin{verbatim}
    for u in V, d[u] = INTEGER_MAX, p[u] = 0, F = V
    d[s] = 0
    while F != Q:
        f[u] = d[u] + h[u]
        u = (vertex in F with min f[u])
        remove u from F
        for v in Neighbor(u):
            if d[u] + h[v] < d[v]:
                p[v] = u
                d[v] = d[u] + h[v]
\end{verbatim}
\begin{itemize}
	\item Note that $F$ is a heap, which allows for quick removal of the minimum item.
\end{itemize}
\subsection*{Heuristics}
What do we choose the \texttt{h[]} function to be?  We chose distance in this case, but what else may be good?
\begin{itemize}
	\item \textbf{Theorem:} A$^*$ is guaranteed to find the optimal route if $h(u)$ is \textit{admissable}.
	\item In this case, the best heuristic we can come up with is the time to travel from one city to another, since that instantly finds a solution.
	\item The worst heuristic we can come up with is $h(u) = 0$, since that would cause us to explore every node just like Dijkstra's.
	\item In other words, the heuristic is always an \textit{underestimate} of the correct solution.
	\begin{itemize}
	    \item If it wasn't an underestimate, it may cause us to miss the optimal path!
    \end{itemize}
    \item We want to show that with a good enough heuristic, A* would find the optimal route, even without exploring every node.
\end{itemize}

\subsubsection*{Proof}
Let P* be the optimal path from point A to point B.  This path includes nodes $w_0, \dots, w_m$, with $w_0$ also being the start, and $w_m$ is the last node before the end, $t$.
\begin{itemize}
	\item \textbf{Lemma:} When each node $w_0$ to $w_{m - 1}$ was last removed from the heap, $d(w_1) = d^*(w_1)$
	\item This is provable by induction.
	\begin{itemize}
	    \item Base: $d[s] - d[w_0] = 0$
	    \item Hypothesis: We assume that every next node is also on the optimal path.  Therefore,  
	    \begin{align*}
            d(w_{k + 1}) &= d(w_k) + len(w_k, w_{k + 1})\\
            &= d^*(w_k) + len\\
            &= d^*(w_{k + 1})
        \end{align*}
    \end{itemize}
\end{itemize}
Suppose for the purpose of contradiction, the A* algorithm found a non-optimal solution.  Let $w_m$ be a node that we decided to visit over $t$, the ending node.
\begin{align*}
    f[t] &= d[t] + h[t]\\
    &= d[t]\\
    &\leq f[w_m]\\
    &= d[w_m] + h[w_m]\\
    &= d^*[w_m] + h[w_m]\\
    &\leq d^*[w_m] + h^*[w_m]\\
    &=length(p^*)
\end{align*}
$p^*$ is the known optimal solution, and $h^*$ is the optimal heuristic.
\begin{itemize}
    \item Remember that $h[t]$ is just 0, since it is the ending node.
	\item Since we visited $w_m$ instead of $t$, that means $f[w_m]$ must have been lower, hence step 3.
	\item $d^*[w_m]$ gives the same value as $d[w_m]$, since we assume the path we took to get there was optimal.  This is based on the lemma.
	\item We know that $h[w_m] \leq h^*[w_m]$, since the heuristic is an underestimate.  $h[w_m]$ was an estimate from $w_m$ to $t$, $h^*[w_m]$ was the actual distance.
	\item Since the sum of the optimal distance plus the optimal heuristic is just the length of the optimal path, this path through $w_m$ must be optimal.  However, this is a contradiction since this path was defined as non-optimal.
	
\end{itemize}

\section*{Bellman-Ford Algortihm}
What if our graph contains negatively weighted edges?  Now Dijkstra's doesn't work since for Dijkstra's, once a node is visited, it is not visited again.\\\\
However, with the introduction of negatively weighted edges, we have a problem to address first:
\textbf{Negative Weight Cycles}
\begin{itemize}
	\item What if a cycle on a graph has a negative weight overall?  In this case, the solution would be infinitely long since it is always "worth it" to go around the cycle one more time.
	\item To fix this, we need to first detect the negative weight cycles before attempting to solve for the lowest weight path between two points.
\end{itemize}
Now, we can start.  The problem statement: Given a directed graph with edge weights $(u, v) \in \mathbb{R}$:
\begin{enumerate}
	\item Determine if it contains a negative cycle
	\item If so, return infinity.  If not, find the shortest $s \rightarrow t$ path.
\end{enumerate}
The algorithm goes as follows:
\begin{itemize}
	\item Let $d_s$[u] be the current estimated $s \rightarrow ud$
	\item d[s] = 0, d[u] = $\infty \forall u \neq s$ 
	\item \textbf{Ford Step:} Find edge $(u, v)$ such that $d_s$[v] > $d_s$[u] + d(u, v).  Set $d_s$[v] = $d_s$[u] + d(u, v).
\end{itemize}


\subsection*{Theorem}
If you cannot relax (via Ford), then $d_s$[u] is the shortest $s \rightarrow u$ path $\forall u$

\subsubsection*{Proof}
We first need some lemmas:\\\\
\underline{Lemma 1:}  After Step $i$, either for any v $\in$ V, $d_s$[v] = $\infty$, or there exists an $s \rightarrow v$ path of length $d_s$[v].\\\\
To prove this lemma, we can use induction.  First, if $d_s$[v] = $\infty$, we can't say anything.  For the other case:  
\begin{itemize}
    \item When $i = 0$ (base case), $d_s$[s] = 0 (which is less tha infinity), there exists an $s \rightarrow s$ path of weight 0.  
    \item Let's assume (inductive step) the lemma is true for $i - 1$.  In the $i$-th step, we update edge $(u, v)$ where $d_s$[v] = $d_s$[u] + d(u, v).
	\item From here, we know that since $d_s$[u] was updated at step $i - 1$ or earlier, there exists a path from $s \rightarrow u$ of length $d_s$[u].
	\item Therefore, there exists a path $s \rightarrow u \rightarrow v$ of distance $d_s$[u] + d(u, v).
\end{itemize}
\underline{Lemma 2:}  When no more Ford steps are possible, for all paths $P_{sv}$, $s \rightarrow v$, length($P_{sv}$) $\geq d_s$[v]\\\\
Thie lemma can also be proven by induction:
\begin{itemize}
	\item When $|P_{sv}| = 1$ (base case), then the path is a single edge.  Therefore, it cannot be relaxed.  $d_s$[v] $\leq$ d(s, v).
	\item Assume the lemma is true for all $|P_{sv}| \leq (k - 1)$.  (Inductive case).  
    \item Let $P_{sv}$ be an $s \rightarrow v$ path of length $k$ edges.  $P_{sv} = P_{su} + (u, v)$  (Note that $P_{su}$ has length 1.)
    \item cost($P_{sv}$) = cost($P_{su}$) + d(u, v)
    \item Inductive Hypothesis $\geq d_s$[u] + d(u, v) $\geq d_s$[v].
\end{itemize}

\subsection*{Implementation}
\begin{verbatim}
Initialize d[u] = infinity forall u, d[s] = 0
for i in [n]:
    for (u, v) in E:
        if d[v] > d[u] + d(u, v):
            d[v] = d[u] + d(u, v)
            parent[v] = u
\end{verbatim}



\section*{Shortest Paths Summary}
\begin{itemize}
	\item BFS, O(n + m), unweighted graphs
	\item Dijkstra, O(m log(n)), positive edge weights, single source all paths
	\item A-Star, O(m log(n)), needs heuristic
	\item Bellman-Ford, O(mn), arbitrary large weights (incl. negatives)
\end{itemize}






\end{document}