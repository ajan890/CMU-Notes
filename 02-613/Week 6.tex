% document formatting
\documentclass[10pt]{article}
\usepackage[utf8]{inputenc}
\usepackage[left=1in,right=1in,top=1in,bottom=1in]{geometry}
\usepackage[T1]{fontenc}
\usepackage{xcolor}

% math symbols, etc.
\usepackage{amsmath, amsfonts, amssymb, amsthm}

% lists
\usepackage{enumerate}

% images
\usepackage{graphicx} % for images
\usepackage{tikz}

% code blocks
% \usepackage{minted, listings} 

% verbatim greek
\usepackage{alphabeta}

\graphicspath{{./assets/images/Week 6}}

\title{02-613 Week 6 \\ \large{Algorithms and Advanced Data Structures}}
 
\author{Aidan Jan}

\date{\today}

\begin{document}
\maketitle

\section*{Divide and Conquer}
\textbf{Divide and Conquer} is a class of algorithms that involve the principle of cutting a problem into smaller parts, solving the smaller instance of the problem, and pieceing it back together.

\subsection*{Merge Sort}
One of the most-studied algorithms that fall into the divide and conquer category is merge sort, a sorting algorithm that essentially goes as follows:
\begin{itemize}
    \item Cut the array in half
    \item Recursively sort the left half and the right half
    \item Put the array back together.
\end{itemize}
\begin{verbatim}
MergeSort(L):
    if |L| = 1: return L
    elif |L| = 2: return [min(L), max(L)]
    else:
        L1 = MergeSort(L[0: |L| / 2])
        L2 = MergeSort(L[|L| / 2: |L|])
        return combine(L1, L2)
\end{verbatim}
The runtime of this algorithm is O($n \log n$).

\subsubsection*{Runtime Proof}
We can prove the runtime of merge sort (like most other divide and conquer algorithms) by mathematical induction. 
\begin{enumerate}
    \item Show $T(k) \leq f(k)$ for "small" $k$
    \item Assume $T(k) \leq f(k)$ for $k < n$
    \item Show $T(n) \leq f(n)$
\end{enumerate}
For step 1, if $k = 2$, then $c \cdot 2 \log 2 = 2c \leq T(n)$, where $c$ is a constant.  Now, assuming $k = n$,
\begin{align*}
    T(n) &\leq 2 T\left(\frac{n}{2}\right) + cn\\
    &\leq 2\left(c \frac{n}{2} \log\left(\frac{n}{2}\right)\right) + cn\\
    &\leq cn \log\left(\frac{n}{2}\right) + cn\\
    &\leq cn \log n - cn \log 2 + 2cn\\
    &\leq cn \log n
\end{align*}
It turns out this math would also work out for any running time larger than $n \log n$, such as $n^2$.  However, we only care about the tightest runtime we can assign.  It will not work for O($n$).
\begin{itemize}
    \item Recall that O($n \log n$) is actually also O($n^2$).
\end{itemize}

\subsection*{The Master Theorem}
For divide and conquer cases, we can typically write the runtime of the worst case as 
\[T(n) = a T\left(\frac{n}{b}\right) + f(n)\]
where $a$ is the number of subproblems the main problem gets split into per iteration, and $b$ is the size of the subproblem relative to the original.  For example for merge sort, $a = 2$ and $b = 2$, since every iteration splits into 2 subproblems, each half the size of the original.  Finally, $f(n)$ is the runtime of the combine operation.  For merge sort, $f(n) = n$.\\\\
The master theorem states that:
\begin{itemize}
    \item If $n^{\log_b a} > f(n)$, then the final runtime would be $n^{\log_b a}$.
    \item If $f(n) > n^{\log_b a}$, then the final runtime would be $f(n)$.
    \item If $f(n) = n^{\log_b a}$, then the final runtime would be $f(n) \log (n^{\log_b a})$.
\end{itemize}

\subsection*{Closest Two Points on a Plane}
Now a more practical application of divide and conquer: given a set of points on a 2D plane, find the pair of points that is closest together.
\begin{itemize}
    \item The naive way is to calculate the distance between every pair of points.  This obviously runs in O($n^2$).  That's slow.
    \item It's possible to solve this problem in O($n \log n$) using a divide and conquer method.
\end{itemize}
The optimal algorithm goes as follows:
\begin{enumerate}
    \item Sort the points by $x$-coordinate
    \item Draw a line down the middle of the points, so that half the points are on the left of the split, and the other half are on the right.
    \item Solve the smallest distance between two points in each half.  (Let this distance be $d$.)
    \item To combine, take the smaller distance from within each half, and check points over the boundary to see if there are two points closer together.
\end{enumerate}
How do we combine?
\begin{itemize}
    \item Since the smallest distance on either half is $d$, we only have to check points within $d$ of our boundary of each side.  From here, consider every point within $d$ from the boundary on the left half, $l_i$.
    \item Record the $y$-coordinate of $l_i$, and create a $2 \times 4$ grid of squares on the right side, centered at the $y$-coordinate.  Let each square have a side length of exactly $d / 2$.
    \begin{center}
        \begin{tikzpicture}
            \draw[dotted] (0, -1) -- (0, 5);
            \draw (0, 0) rectangle (1, 1);
            \draw (1, 0) rectangle (2, 1);
            \draw (0, 1) rectangle (1, 2);
            \draw (1, 1) rectangle (2, 2);
            \draw (0, 2) rectangle (1, 3);
            \draw (1, 2) rectangle (2, 3);
            \draw (0, 3) rectangle (1, 4);
            \draw (1, 3) rectangle (2, 4);
            \fill (-0.5, 2) circle (2pt) node[below left] {$l_i$};;
        \end{tikzpicture}
    \end{center}
    \item Importantly, since each square has side length $d / 2$ (and therefore a square diagonal is $\sqrt{d / 2}$), there must be at maximum one point in each square.  If there is more, we would have found that pair and marked it as the smallest, which would contradict our choice of $d$.
    \item Now, for each point near the divide, we need to check a constant number (8) possible points maximum on the other side of the line.
    \item Because of this constant checking, we can now assume that this combine function runs worst case on $n$ points, and O(1) per point.
    \item This gives a divide and conquer runtime of O($n\log n$) in total!
\end{itemize}

\subsection*{Multiplication}
Most intuitively, multiplication is repeated addition.  (For computers, this would occur in the binary space.)  Doing this with a $n$-bit number would require O($n^2$) time.  However, there is a faster way.\\\\
Say $x$ is an $n$-bit binary number.  Let $m = n / 2$.  $x_0$ are the lower $m$ bits, and $x_1$ is the upper $m$ bits.\\\\
In this case,
\begin{align*}
    x &= x_1 2^m + x_0\\
    y &= y_1 2^m + y_0
\end{align*}
Now, when we multiply these two numbers, we have
\begin{align*}
    xy &= (x_1 2^m + x_0)(y_1 2^m + y_0)\\
    &= x_1 y_1 2^{2m} + (x_1 y_0 + x_0 y_1) 2^m + x_0 y_0
\end{align*}
We can then expand the middle term even more:
\begin{align*}
    x_1 y_0 + x_0 y_1 = (x_1 + x_0)(y_1 + y_0) - x_1 y_1 - x_0 y_0
\end{align*}
Why does this help?  We just turned 2 multiplications into three!
\begin{itemize}
	\item However, notice that the parts of these multiplications have already been used.  This actually turns two multiplication into one, as long as we cache from the steps before!
\end{itemize}

\subsubsection*{Proof}
\[T(2^k) = 3T(2^{k - 1}) + c 2^k\]
where $T$ is the number of elementary operations, and $k$ is the number of bits of the number.  To multiply $k$-bit numbers, since we do three multiplications of two numbers half as large and a multiplication of $2^k$ (bit shift), we get the above equation.  Simplifying,
\begin{align*}
    \frac{T(2^k)}{3^k} &= \frac{3T(2^{k - 1})}{3^k} + \frac{c 2^k}{3^k}\\
    \frac{T(2^k)}{3^k} &= \frac{T(2^{k - 1})}{3^{k - 1}} + \frac{c 2^k}{3^k}\\
    \frac{T(2^k)}{3^k} &= \left(\frac{T(2^{k - 2})}{3^{k - 2}} + \frac{c 2^{k - 1}}{3^{k - 1}}\right) + \frac{c 2^k}{3^k}
\end{align*}
We can keep breaking down the time of the middle term using mathematical induction.  (We know the base case is true, since $T(1) = 1$.)  This means this equation can be simplified to:
\[\frac{T(1)}{3} + c \sum_{i = 2}^k \frac{2^i}{3^i}\]
Now, since $\frac{2^i}{3^i}$ will eventually converge to some number (since its rate is less than 1), we can assign a boundary, $\beta$, that this sum must be under.  We can then rewrite this in terms of $\beta$:
\begin{align*}
    \frac{T(2^k)}{3^k} &\leq \beta\\
    T(2^k) &\leq \beta 3^k\\
    &= \beta 2^{\log 3^k} \\
    &= \beta 2^k \log 3 \\
    &= \beta n^{\log 3}
\end{align*}
Since beta is a constant, we can drop it, and say our time complexity is $\Theta(n^{\log 3}) = \Theta(n^{1.58})$.
\begin{itemize}
	\item Note that $\log$ here is in base-2.
\end{itemize}

\subsection*{Matrix Multiplication}
Note that we can do matrix multiplication in a similar way to addition.
\begin{itemize}
	\item Break our $n \times n$ matrices into four $\frac{n}{2} \times \frac{n}{2}$ matrices.  We now have 8 small matrices total.
	\[A \times B = \begin{bmatrix} A_{11} & A_{12} \\ A_{21} & A_{22} \end{bmatrix} \begin{bmatrix} B_{11} & B_{12} \\ B_{21} & B_{22} \end{bmatrix}\]
    \item We can calculate the result of these 8 multiplications in 7, if we do some smart grouping.
    \item Our runtime for a $n \times n$ matrix becomes $T(n) = 7 T(n / 2) + c n^2$.
    \item Final runtime after simplifying using the same math gives O($n^{\log 7}$)
\end{itemize}

\subsection*{Binary Search}
Normally if we want to search for an object in a data set, it takes O($n$), since we have to check every element.  Can we do better?
\begin{itemize}
	\item We can do better.  Set up a binary search tree.  First, set a (arbitrary) node to be the root.
	\item From the root, when a number is inserted, if it is smaller in the root, append to the left subtree.  If larger, then append to the right.
	\item \textbf{BST Property:} Given a node ($k, v$), all nodes in left subtree have keys $< k$, and all nodes in right subtree have keys $> k$.
	\item To find a node, start from the root.  If the value you are looking for is lower, go left.  If the value is larger, then look right.
	\item To insert, follow the same process as find except when you get to the end, append the node.
	\item To delete, find the node to delete, $u$, and mark its parent $p$.
	\begin{itemize}
	    \item if $u$ is a leaf, just delete,
	    \item if $u$ has 1 child $c$, delete $u$, and make $c$ a child of $p$.
	    \item if $u$ has 2 children, find the smallest node in the right subtree of $u$, delete it, and replace $u$ with it.
    \end{itemize}
\end{itemize}
The issue is, what if our root is very small, and all the nodes we append are bigger?  Then, the tree grows only to one side, and eventually turns into a linked list.
\begin{itemize}
	\item This makes the search time O($n$) instead of O($\log n$)!
	\item Therefore, we occasionally have to rebalance the tree and pay the O($n$) time.  (Read each value in order of the tree, which gives you sorted list, pick the median and set it as root, and reinsert all the other values.)
\end{itemize}

\subsection*{Averaging}
We have two types of averages:
\begin{itemize}
	\item \textbf{Randomized average:} Suppose we roll a fair 6-sided dice.  The average value should be 3.5.  However, the average could also be 6, albeit very unlikely.
	\item \textbf{Amortized average:} Suppose we have a deck of cards, numbers 1-10 with 4 repeats each.  If we sample 40 cards without replacement, the guaranteed average is 5.5.  There is no chance of anything else.
	\begin{itemize}
	    \item Amortized averages have strong restraints on what the numbers can be.
	    \item This is commonly used in video games because it oftentimes feels more fair than full randomness.
    \end{itemize}
\end{itemize}

\subsubsection*{QuickSort}
\begin{itemize}
	\item Pick a random number as a "pivot", and sort each side into whether it is smaller than or larger than the pivot.  The sides need not be balanced.
	\item Recursively split each half, and combine them as you go up.
    \item In this case, \textit{on average}, the tree will be mostly balanced, since on average half the numbers will be bigger, and half will be smaller.
    \item This is a form of amortized average!
\end{itemize}


\end{document}
