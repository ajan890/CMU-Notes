% document formatting
\documentclass[10pt]{article}
\usepackage[utf8]{inputenc}
\usepackage[left=1in,right=1in,top=1in,bottom=1in]{geometry}
\usepackage[T1]{fontenc}
\usepackage{xcolor}
\usepackage[table,xcdraw]{xcolor}

% math symbols, etc.
\usepackage{amsmath, amsfonts, amssymb, amsthm}

% lists
\usepackage{enumerate}
\usepackage{tabularx}

% images
\usepackage{graphicx} % for images
\usepackage{tikz}

% code blocks
\usepackage{minted, listings} 

% verbatim greek
\usepackage{alphabeta}

\graphicspath{{./assets/images/Week 4}}

\title{03-621 Week 4 \\ \large{Advanced Quantative Genetics}}
 
\author{Aidan Jan}

\date{\today}

\begin{document}
\maketitle

\section*{Analysis of Single Gene Traits}
Every human chromosome contains \textbf{at least} one gene that is defective in a monogenic disorder.
\begin{itemize}
	\item Although single-gene disorders \textit{individually} are relatively rare, \textit{collectively} they are important contributors to human disease.
\end{itemize}
Single gene traits are also known as ``Monogenic Traits / Disorders''.  Traits for which different alleles at a single gene are sufficient to specify different phenotypes.
\begin{itemize}
	\item There may be more than two phenotypes
	\item There may be more than two alleles
	\item Dominance may be complete or incomplete, or there may be co-dominance.
\end{itemize}
Note: there may be \textit{more than one gene} that affects the trait and whose alleles can produce similar or related phenotypes.\\\\
Some of the most common single gene disorders in humans include:
\begin{itemize}
	\item Thalassemia (Chromosome 16/11), reduced amounts of hemoglobin; anemia, bone and spleen enlargement.  Affects 10\% of Italy.
	\item Sickle-cell anemia (Chromosome 11), abnormal hemoglobin; sickle-shaped red cells, anemia, blocked circulation, increased resistance to malaria.  Affects $1/625$ African-Americans
	\item Hypercholesterolemia (Chromosome 19), missing protein that removes cholesterol from the blood, heart attack by age 50.  Affects $1/122$ of French Canadians.
\end{itemize}
To calculate progeny risk, we need to know (or estimate) the genotypes of the parents
\begin{itemize}
	\item Even if relevant gene(s) are known, not all the possible disease-causing alleles (which individually may be rare or new mutations) are identified, so molecular diagnostics are not always helpful.
	\item We can't do intentional test crosses to figure out the genotypes of individuals with dominant phenotypes\dots but sometimes they appear in pedirees.
\end{itemize}

\subsection*{Mendelian Traits}
\begin{itemize}
	\item Discrete binary phenotypes
	\item Two alleles at a single gene
	\item Complete dominance
	\item No effect of environment
	\item Full penetrance and expressivity
	\item No sex-linkage
	\item Not sex-limited
	\item No parent-of-origin effect.
\end{itemize}
These traits are described completely by Mendel's Law of Equal Segregation, and can be drawn into Punnett squares with expected ratios.\\\\
\textbf{If mating is random} and there are \textbf{two alleles} at a given locus with frequencies $p$ and $q$, the distribution of genotype frequencies \textit{in the population} is given by the binomial expression:
\[1 = (p + q)^2 = p^2 + 2pq + q^2\]
This is known as the \textbf{Hardy Weinberg Principle}.

\subsection*{A Problem in Human Genetics}
Human genetics have the issue of small progeny numbers.
\begin{itemize}
	\item In Mendel's Pea Experiment, he crossed round x wrinkled peas, and counted the F1 and F2 progeny.  There were 5474 round peas and 1850 wrinkled peas in F2.  (nearly perfectly 3:1 ratio)
	\item Suppose we have two humans heterozygous for Cystic fibrosis (for reference, Cystic fibrosis occurs with two recessive alleles.)  We only get 4 children.  There's a $0.75^4 \approx 0.32$ chance that none of them will exhibit the disease at all.
\end{itemize}

\subsection*{Analysis of Single Gene Traits using Human Pedigrees}
\begin{enumerate}
	\item Infer the mode of inheritance from pedigree patterns
	\begin{itemize}
	    \item Dominant/Recessive
	    \item Autosomal/Sex chromosome-linked
	    \begin{itemize}
	        \item Map chromosomal location, identify gene, and run diagnostic tests/treatments
        \end{itemize}
    \end{itemize}
    \item Genetic counseling: How to estimate risk
    \begin{itemize}
	    \item Probability of carrier status in parent
	    \item Risk of disease in progeny
    \end{itemize}
\end{enumerate}

\subsection*{Reading Human Pedigrees}
\begin{center} 
	\includegraphics*[width=0.8\textwidth]{W4_1.png} 
\end{center}
Consider these human pedigrees for congenital deafness (an autosomal recessive trait):
\begin{center} 
	\includegraphics*[width=0.8\textwidth]{W4_2.png} 
\end{center}
How do we know the trait is recessive?
\begin{itemize}
	\item If the trait were dominant, then the parents in generation I must have the trait!
\end{itemize}
How come the children on generation III do not have the trait, if both parents do?
\begin{itemize}
	\item The parents must be in different complementation groups!
	\item The offspring of generation III are all double heterozygotes.
\end{itemize}

\subsection*{How to Analyze Pedigrees}
\begin{itemize}
	\item Identify the mode of inheritance from patterns in the pedigree:
	\begin{itemize}
	    \item Recessive autosomal traits
	    \item Dominant autosomal traits
	    \item Recessive X-linked traits
	    \item Dominant X-linked traits (few known)
	    \item Y-linked traits (very few known)
    \end{itemize}
    \item It is easier to \textbf{rule out} a mode than to prove one conclusively
    \item When two or more modes cannot be ruled out, we must assess their \textbf{relative probabilities}
\end{itemize}

\subsection*{Frequent Features}
\subsubsection*{Autosomal Recessive Disorders}
\begin{enumerate}
	\item Affected progeny can have two unaffected parents (frequent for rare traits)
	\item Males and females are affected in equal proportion
	\item Male-to-male transmission can occur
	\item Recessive traits are more commonly expressed in \textbf{inbred} familes or populations
\end{enumerate}
Recessive traits are more commonly expressed in \textbf{inbred} familes or populations.
\begin{itemize}
	\item If one common ancestor carries a recessive allele, then at least some of the children in II would be heterozygotes.
	\item If inbreeding occurs in any future generation with two heterozygotes, their children would have a chance of expressing the recessive trait.
\end{itemize}

\subsubsection*{Autosomal Dominant Disorders}
\begin{enumerate}
	\item Affected individuals always have an affected parent (mother or father, except when a mutation first arises)
	\item Phenotype frequently appears in every generation (recessives can skip)
	\item On average 50\% of progeny are affected
    \item Trait can be transmitted from both males and females to both sons and daughters
\end{enumerate}
Some \textbf{dominant lethal} traits can be transmitted because the phenotype does not manifest before reproductive age.

\subsubsection*{X-linked Recessive Disorders}
\begin{enumerate}
	\item Phenotype appears much more frequently in males than females
	\item All daughters of an affected male are ``carriers'' (unaffected heterozygotes)
	\item In the next generation, half the sons of these carriers show the phenotype
	\item An affected female always has an affected father
	\item Reciprocal crosses (switched sexes) give different results
\end{enumerate}
For example, red-green colorblindness.

\subsubsection*{X-linked Dominant Disorders}
\begin{enumerate}
	\item Affected individuals always have an affected parent
	\item Affected males pass the condition to ALL their daughters, but not their sons
	\item Affected heterozygous females pass the condition to HALF their daughters AND sons
	\item On average, more affected females than males (consequence of (2) and (3))
\end{enumerate}
(2, 3, 4 are unique, but can only be assessed via probabilities of pedigrees).  There are few known examples, one of which is brown tooth enamel.

\subsubsection*{Y-linked Disorders}
Y-linked disorders are transmitted exclusively from fathers to sons.  For example, hairy ear rims.

\subsection*{Calculating Risks in Pedigree Analysis}
What is the probability that a child has X disease?
\begin{itemize}
	\item Usually, the type of disease (e.g., autosomal recessive) will be given.
	\item If the disease is ``rare'', then it can be assumed that people sampled from the population would be homozygous for the allele without the disease.
	\item We may have to write probabilities of genotypes for multiple generations before the child we care about, since we might not know their genotypes either.
\end{itemize}

\subsection*{Bayesian Analysis to Account for Conditional Information}
\begin{enumerate}
	\item Identify all the different scenarios that can explain the observations
	\item For each scenario, calculate the \textbf{prior probability} and \textbf{conditional probability}
	\item Multiply the prior probability by the conditional probably to obtain a \textbf{joint probability} for each scenario
	\item Determine what fraction of the total joint probability is represented by each individual scenario, to get a \textbf{posterior probability} for each scenario.
\end{enumerate}

\section*{Variable Expression of Nucleus-Encoded Phenotypes}
\begin{itemize}
	\item \textbf{Penetrance:} \% of individuals with a particular mutant \underline{genotype} who display the corresponding \underline{phenotype}
	\begin{itemize}
	    \item 100\% = complete penetrance
	    \item <100\% = incomplete penetrance
    \end{itemize}
    \item \textbf{Expressivity:} degree of \underline{severity} with which a given phenotype is expressed in an individual with the corresponding genotype
\end{itemize}
\begin{center} 
	\includegraphics*[width=0.6\textwidth]{W4_3.png} 
\end{center}

\subsection*{X-Inactivation}
Inactivation of one X chromosome in females equalizes X-linked gene expression in XX and XY genotypes.  (``dosage compensation'')
\begin{itemize}
	\item The inactive X is recognized cytologically as a condensed ``Barr body''.
	\item Heterozygous females may display a recessive trait or have a milder version of a dominant trait than affected males because half of the time the gene would be unactivated.
	\begin{itemize}
	    \item As a consequence, some traits may be exclusively heterozygous (e.g., Dwarfism), since the homozygous genotype is lethal.
    \end{itemize}
\end{itemize}
A heterozygous female is a \textbf{genetic mosaic} containing cell clones that express one or the other allele of X-linked genes.
\begin{itemize}
	\item An example of this is calico cats.  The different colors of fur are caused by different alleles being expressed.
	\item 99\% of calico cats are female.  The remaining 1\% male calico cats are caused by the rare XXY genotype.
\end{itemize}
\end{document}