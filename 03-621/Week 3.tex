% document formatting
\documentclass[10pt]{article}
\usepackage[utf8]{inputenc}
\usepackage[left=1in,right=1in,top=1in,bottom=1in]{geometry}
\usepackage[T1]{fontenc}
\usepackage{xcolor}
\usepackage[table,xcdraw]{xcolor}

% math symbols, etc.
\usepackage{amsmath, amsfonts, amssymb, amsthm}

% lists
\usepackage{enumerate}
\usepackage{tabularx}

% images
\usepackage{graphicx} % for images
\usepackage{tikz}

% code blocks
\usepackage{minted, listings} 

% verbatim greek
\usepackage{alphabeta}

\graphicspath{{./assets/images/Week 3}}

\title{03-621 Week 3 \\ \large{Advanced Quantative Genetics}}
 
\author{Aidan Jan}

\date{\today}

\begin{document}
\maketitle

\section*{Mutations (Continued)}
\begin{itemize}
	\item Synonymous Substitutions Can Alter Phenotypes
	\begin{center} 
        \includegraphics*[width=0.7\textwidth]{W3_1.png} 
    \end{center}
    \item Single-gene disorders can also be caused by mutations in \textbf{non-coding} RNA genes
    \item Large Copy Number Variations (CNVs) are common in the human genome
    \begin{itemize}
        \item One section of DNA is duplicated many times
        \item About 10-15\% of the genome displays copy number variation
        \item 1000 bp - 5 mbp in length
        \item Most are in non-coding regions, but some contain genes
        \item Too large for PCR analysis, can be detected using a microarray
    \end{itemize}
    \item Variable nucleotide tandem repeats (VNTR) and short tandem repeats (STR) are inherited repeating stretches of DNA
    \begin{itemize}
        \item Different individuals may have different numbers of repeats at a given locus.
        \item If the number of base pairs in a repeated section is not a multiple of three, tandem repeats may cause frameshift errors.
    \end{itemize}
    \item Some mutations that cause disease do not change the sequence of a gene product but alter the amount of gene product
    \begin{itemize}
        \item For example, mutations in promotors or enhancers
        \item Mutations in transfactors (e.g., histones)
        \item Gene duplication or deletion
    \end{itemize}
    \begin{center} 
        \includegraphics*[width=0.8\textwidth]{W3_2.png} 
    \end{center}
    \item Gene dosage changes via Meiotic Non-disjunction
    \begin{itemize}
        \item Frequency: in all recognized pregnancies
        \item 8\% of \textit{recognized} pregnancies have major chromosomal abnormalities.  $>94\%$ of this $8\%$ undergo spontaneous abortion.
        \item Aneuplodies change the amount of gene product expressed ("gene dosage"), across many genes.
        \item Among the fetuses in 100000 recognized pregnancies, about 8000 have major chromosomal abnormalities, 7500 of these undergo spontaneous abortion, and 500 are born alive.
    \end{itemize}
    \begin{center} 
        \includegraphics*[width=0.8\textwidth]{W3_3.png} 
    \end{center}
\end{itemize}





\end{document}